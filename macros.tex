% Makra używane w projekcie

% Proste umieszczenie obrazu
\newcommand{\putcaptedimage}[3]
{
	\begin{figure}[h]
		\begin{center}
			\includegraphics[width=\textwidth]{#1}
			\caption{#2}
			\label{#3}
		\end{center}
	\end{figure}
}

% Proste umieszczenie tabelki
\newcommand{\puttable}[2]
{
	\begin{center}
		\begin{tabular}{#1}
			#2
		\end{tabular}
	\end{center}
}

% Karta wroga
\newcommand{\larpcardlineattack}[2]
{
	\textbf{Atak z siłą #2}: #1
}
\newcommand{\larpcardenemy}[6]
{
	\begin{tcolorbox}[colback=white]
		\begin{center}
			\textbf{#1}
		\end{center}

		#2
		
		#3
		
		\puttable{c c c}{
		\textbf{Test:} $\geqslant$ #4 & \textbf{Życie:} #5 & \textbf{Krew:} #6 \\
		}
		
	\end{tcolorbox}
}

% Karta obiektu
\newcommand{\larpcarditem}[2]
{
	\begin{tcolorbox}[colback=white]
		\begin{center}
			\textbf{#1}
		\end{center}
		\begin{center}
		#2
		\end{center}
		
	\end{tcolorbox}
}

% Karta bohatera
\newcommand{\larpcardlinespecies}[3]{
	\puttable{l r}{
		\textbf{Gatunek:} #1 & \textbf{Pozycja:} #2 \\ \multicolumn{2}{c}{\textbf{Przynależność:} #3} \\
	}
}
\newcounter{counterattr}
\newcommand{\larpcardlineattributes}[3]{
	\puttable{l c r}{
		\textbf{Siła:} #1 & \textbf{Siła Woli:}
			\setcounter{counterattr}{0}
			\loop
			\Square
			\addtocounter{counterattr}{1}\ifnum\value{counterattr}<#2
			\repeat
		& \textbf{Życie:}
			\setcounter{counterattr}{0}
			\loop
			\Square
			\addtocounter{counterattr}{1}\ifnum\value{counterattr}<#3
			\repeat
		\\
	}
}
\newcommand{\larpcardlinepower}[4]{
	#1 -- \textbf{#2} -- #3 -- #4
}
\newcommand{\larpcardlineversus}[2]{
	#1 $\nleftrightarrow$ #2
}
\newcommand{\larpcardlinepassive}[1]{
	Pasywna \textbf{Test wroga:} $<$ #1
}
\newcommand{\larpcardcharacter}[7]
{
	\begin{tcolorbox}[colback=white]
		\begin{center}
			\textbf{#1}
		\end{center}
		#2
		
		#3
		
		\begin{center}
			\textbf{Moce:}
		\end{center}
		
		#4
		
		\begin{center}
			\textbf{Cele:}
		\end{center}
		#5
		
		#6
		
		\begin{center}
			\textbf{Zasady:}
		\end{center}
		{
		\tiny
		#7
		}
		
	\end{tcolorbox}
}

% Mechaniki wspólne
\newcommand{\larplistall}{
	\begin{itemize}[noitemsep]
		\item Test cechy -- rzuć monetami w liczbie cechy, musisz wyrzucić liczbę sukcesów $\geqslant$ jak wróg.
		\item Obrażenia otrzymuje atakowana strona w liczbie różnic sukcesów.
		\item Liczniki Siły Woli i Życia są w formie trójwartościowych wskaźników. Brak obrażeń, obrażenie lekkie i ciężkie.
		\item Możesz przerzucić zwykłą monetę za powierzchowne obrażenie Siły Woli.
		\item Test Siły Woli -- rzuć monetami w liczbie pustych kratek Siły Woli.
		\item Możesz oddać swoje punkty życia wampirowi w letargu, aby go obudzić.
	\end{itemize}
}

% Mechaniki użytkowników krwi
\newcommand{\larplistblood}{
	\begin{itemize}[noitemsep]
		\item Ilość monet krwi oznacza poziom krwi.
		\item Pobudzenie krwi polega na rzucie jedną monetą krwi i stracie jej przy czarnym wyniku.
		\item Pobudź krew, aby wyleczyć 1 powierzchownych obrażeń.
		\item Pobudź krew, aby wzmocnić się na jeden test o 1 punktów siły, można powtarzać.
		\item Przy użyciu każdej dyscypliny pobudź krew.
	\end{itemize}
}
	
% Mechaniki wampirów
\newcommand{\larplistvampire}{
	\begin{itemize}[noitemsep]
		\item Po utracie życia zapadasz w letarg i ktoś musi oddać ci punkty życia, abyś wstał.
		\item Po wyjściu z letargu pobudzasz krew i jeśli się nie uda, wpadasz w szał.
		\item Po utracie krwi twoja Bestia wpada w szał.
		\item Po utracie Siły Woli szalejesz i wpadasz w depresję. Wywołuje się twoja kompulsja klanowa.
		\item Gdy twoja Bestia wpada w szał, to chce kogoś natychmiast wyssać. Twoja siła jest podwojona na ten czas.
	\end{itemize}
}

% Mechaniki śmiertelników
\newcommand{\larplistmortal}{
	\begin{itemize}[noitemsep]
		\item Po utracie życia jesteś w stanie krytycznym na granicy śmierci.
		\item PO utracie Siły Woli wpadasz w depresję.
	\end{itemize}
}

% Mechaniki ghuli
\newcommand{\larplistghoul}{
	\begin{itemize}[noitemsep]
		\item Po utracie krwi nie wpadasz w szał.
		\item Odzyskujesz punkt krwi co 10 minut.
	\end{itemize}
}
