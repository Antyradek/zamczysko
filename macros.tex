% Makra używane w projekcie

% Proste umieszczenie obrazu
\newcommand{\putcaptedimage}[3]
{
	\begin{figure}[h]
		\begin{center}
			\includegraphics[width=\textwidth]{#1}
			\caption{#2}
			\label{#3}
		\end{center}
	\end{figure}
}

% Proste umieszczenie tabelki
\newcommand{\puttable}[2]
{
	\begin{center}
		\begin{tabular}{#1}
			#2
		\end{tabular}
	\end{center}
}

% Karta wroga
\newcommand{\larpcardenemy}[5]
{
	\begin{tcolorbox}[colback=white]
		\begin{center}
			\textbf{#1}
		\end{center}

		#2
		
		\puttable{c c c}{
		\textbf{Siła:} #3 & \textbf{Życie:} #4 & \textbf{Krew:} #5 \\
		}
		
	\end{tcolorbox}
}

% Karta obiektu
\newcommand{\larpcarditem}[2]
{
	\begin{tcolorbox}[colback=white]
		\begin{center}
			\textbf{#1}
		\end{center}
		\begin{center}
		#2
		\end{center}
		
	\end{tcolorbox}
}

% Karta bohatera
\newcommand{\larpcardlinespecies}[3]{
	\puttable{l r}{
		\textbf{Gatunek:} #1 & \textbf{Pozycja:} #2 \\ \multicolumn{2}{c}{\textbf{Przynależność:} #3} \\
	}
}
\newcounter{counterattr}
\newcommand{\larpcardlineattributes}[3]{
	\puttable{l c r}{
		\textbf{Siła:} #1 & \textbf{Siła Woli:}
			\setcounter{counterattr}{0}
			\loop
			\Square
			\addtocounter{counterattr}{1}\ifnum\value{counterattr}<#2
			\repeat
		& \textbf{Życie:}
			\setcounter{counterattr}{0}
			\loop
			\Square
			\addtocounter{counterattr}{1}\ifnum\value{counterattr}<#3
			\repeat
		\\
	}
}
\newcommand{\larpcardlinepower}[5]{
	#1 (#3) -- \textbf{#2} -- #4 -- #5
}
\newcommand{\larpcardlineversus}[2]{
	#1 \emph{versus} #2
}
\newcommand{\larpcardcharacter}[6]
{
	\begin{tcolorbox}[colback=white]
		\begin{center}
			\textbf{#1}
		\end{center}
		#2
		
		#3
		
		\begin{center}
			\textbf{Moce:}
		\end{center}
		
		#4
		
		\begin{center}
			\textbf{Cele:}
		\end{center}
		#5
		
		#6
		
	\end{tcolorbox}
}

% Mechaniki wspólne
\newcommand{\larplistall}{
	\begin{itemize}
		\item Liczniki Siły Woli i Życia są w formie trójwartościowych wskaźników. Brak obrażeń, obrażenie lekkie i ciężkie.
		\item Możesz przerzucić zwykłą monetę za powierzchowne obrażenie Siły Woli.
		\item Test Siły Woli -- rzuć monetami w liczbie pustych kratek Siły Woli.
		\item Możesz oddać swoje punkty życia wampirowi w letargu, aby go obudzić.
	\end{itemize}
}

% Mechaniki użytkowników krwi
\newcommand{\larplistblood}{
	\begin{itemize}
		\item Ilość monet krwi oznacza poziom krwi.
		\item Pobudzenie krwi polega na rzucie jedną monetą krwi i stracie jej przy czarnym wyniku.
		\item Pobudź krew, aby wyleczyć 1 powierzchownych obrażeń.
		\item Pobudź krew, aby wzmocnić się na jeden test o 1 punktów siły, można powtarzać.
	\end{itemize}
}
	
% Mechaniki wampirów
\newcommand{\larplistvampire}{
	\begin{itemize}
		\item Po utracie życia zapadasz w letarg i ktoś musi oddać ci punkty życia, abyś wstał.
		%TODO szał
		
	\end{itemize}

}
