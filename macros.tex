% Makra używane w projekcie

% Proste umieszczenie obrazu
\newcommand{\putcaptedimage}[3]
{
	\begin{figure}[h]
		\begin{center}
			\includegraphics[width=\textwidth]{#1}
			\caption{#2}
			\label{#3}
		\end{center}
	\end{figure}
}

% Proste umieszczenie tabelki
\newcommand{\puttable}[2]
{
	\begin{center}
		\begin{tabular}{#1}
			#2
		\end{tabular}
	\end{center}
}

% Karta wroga
\newcommand{\larpcardlineattack}[2]
{
	\textbf{Atak z siłą #2}: #1
}
\newcommand{\larpcardenemy}[6]
{
	\begin{tcolorbox}[colback=white]
		\begin{center}
			\textbf{#1}
		\end{center}

		#2
		
		#3
		
		\puttable{c c c}{
		\textbf{Test:} $\geqslant$ #4 & \textbf{Życie:} #5 & \textbf{Krew:} #6 \\
		}
		
	\end{tcolorbox}
}

% Karta obiektu
\newcommand{\larpcarditem}[2]
{
	\begin{tcolorbox}[colback=white]
		\begin{center}
			\textbf{#1}
		\end{center}
		\begin{center}
		#2
		\end{center}
		
	\end{tcolorbox}
}

% Karta bohatera
\newcommand{\larpcardlinespecies}[3]{
	\puttable{l r}{
		\textbf{Gatunek:} #1 & \textbf{Pozycja:} #2 \\ \multicolumn{2}{c}{\textbf{Przynależność:} #3} \\
	}
}
\newcounter{counterattr}
\newcommand{\larpcardlineattributes}[3]{
	\puttable{l c r}{
		\textbf{Siła:} #1 & \textbf{Siła Woli:}
			\setcounter{counterattr}{0}
			\loop
			\Square
			\addtocounter{counterattr}{1}\ifnum\value{counterattr}<#2
			\repeat
		& \textbf{Życie:}
			\setcounter{counterattr}{0}
			\loop
			\Square
			\addtocounter{counterattr}{1}\ifnum\value{counterattr}<#3
			\repeat
		\\
	}
}
\newcommand{\larpcardlinepower}[4]{
	#1 -- \textbf{#2} -- #3 -- #4
}
\newcommand{\larpcardlineversus}[2]{
	#1 $\nleftrightarrow$ #2
}
\newcommand{\larpcardlinepassive}[1]{
	Pasywna \textbf{Test wroga:} $<$ #1
}
\newcommand{\larpcardlinecompulsion}[1]{
	\begin{center}
		\textbf{Kompulsja:} #1
	\end{center}
}
\newcommand{\larpcardcharacter}[7]
{
	\begin{tcolorbox}[colback=white, size=fbox]

		\begin{center}
			\textbf{#1}
		\end{center}
		#2
		
		#3
		
% 		\begin{center}
% 			\textbf{Moce:}
% 		\end{center}
		
		#4
		
% 		\begin{center}
% 			\textbf{Cele:}
% 		\end{center}
		#5
		
		#6
		
		#7
	\end{tcolorbox}
}
