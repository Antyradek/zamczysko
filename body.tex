Ten dokument musi być trzymany w tajemnicy przed graczami.
Każdy może mieć wgląd jedynie do własnej karty postaci.

\section{Fabuła}
	Zamek lokalnego Księcia Camarilli płonie.
	Poddani zwabieni łuną pożaru próbują dowiedzieć się co się dzieje.
	Jednakże zamek skrywa wiele tajemnic, i nie jesteście jedynymi zainteresowanymi.
	
	Książę Bożydar Szymonowicki został wciągnięty w czeluście zamku i fajnie było by go uratować.
	
	\subsection{Wstęp fabularny}
	\label{sec:fabularny}
		Ta sekcja powinna być wydrukowana publicznie do zaznajomienia się z każdym graczem.
		\subsubsection{Wampiry}
			W Świecie Mroku wampiry to połowicznie-martwe istoty, które żywią się krwią do przeżycia.
			Zabija je słońce, UV i ogień. Nie stosują się legendy o czosnku, lustrach, liczeniu, zapraszaniu itp.
			Wbicie kołka w serce i pełne obrażenia powodują letarg.
			Posiadają moce regeneracyjne, a po ostatecznej śmierci zmieniają się w popiół.
			
			Społeczeństwo wampirów dzieli się na klany na podstawie linii krwi.
			Geneza wampirów zaczyna się od biblijnego Kaina, przeklętego przez Boga.
			W trzecim pokoleniu powstało 13 klanów, a potem był Potop. Najpopularniejsze z nich:
			\begin{description}[noitemsep]
				\item[Banu Haquim] Daleki klan ze wschodu o którym mało kto słyszał.
				\item[Brujah] Rewolucjoniści i wojownicy z mocami wzmagającymi prędkość i siłę.
				\item[Gangriel] Dzikusy z komunikacją ze zwierzętami i ukrywaniem się.
				\item[Malkawian] Szaleńcy, moce niewidzialności, nadwrażliwi na świat, kontrolują umysły, widzą niestworzone rzeczy.
				\item[Nosferatu] Brzydale, siedzą w piwnicach w łachmanach, dobrzy w technologii, atakują z zaskoczenia.
				\item[Toreador] Piękni i ponętni, uwodzą ofiary, artyści.
				\item[Tremere] Magowie krwi, trzymają się razem i chronią swoich sekretów.
				\item[Tzimisce] Starzy wojewodzi, bezwzględni właściciele ziem, kolektorzy skarbów i wasali, powszechni w Europie.
				\item[Ventrue] Dyrektorzy i bezwzględni korporacjoniści, chcą władzy i przyporządkowania sobie wszystkiego.
			\end{description}
			
			Dyscypliny i moce wampirów:
			\begin{description}[noitemsep]
				\item[Animalizm] Zmiana w zwierzęta i komunikacja z nimi. Ukrywanie się.
				\item[Dominacja] Kontrole umysłu i przymuszenia do działań.
				\item[Nadwrażliwość] Widzenie niewidocznego. Szaleństwo.
				\item[Nekromancja] Tajemnicze moce zaglądania do świata nieumarłych.
				\item[Niewidoczność] Ukrywanie się, podszywanie się, niezwracanie uwagi.
				\item[Odporność] Zwiększenie fizycznej wytrzymałości ciała.
				\item[Potencja] Zwiększenie swojej siły.
				\item[Prezencja] Zastraszanie bądź zauraczanie.
				\item[Przyspieszenie] Zwiększenie swojej prędkości.
				\item[Transformacja] Zmiana kształtów, wtapianie w ściany, zamiana w mgłę.
				\item[Magia Krwi] Sekretna magia strzeżona przez Tremere.
			\end{description}
			
			Maskarada to odwieczna zasada ukrywania swojej tożsamości przed śmiertelnikami.
			Tutaj nie bardzo jest przed kim się ukrywać.
			
			Sekty to globalne organizacje. Camarilla to tradycyjna organizacja chcąca władzy. Anarchiści to wyzwoleni i niezrzeszeni, którzy nie chcą wampirzej władzy. Sabat to złe stowarzyszenie, które morduje kogo popadnie i za nic ma Maskaradę.

			Inkwizycja jest tajną organizacją ludzi, próbującą zwalczyć wszystkie wampiry na Ziemi.
			Wsparcie Watykanu i rządów na świecie.
			Zakon Św. Leopolda działa od czasów średniowiecza.
			Zamek jest ostrzeliwany przez Inkwizycję, a przejścia się czasami zawalają.
			
			Diabolizm polega na wypiciu całej krwi i wyssaniu jednostek naczyń krwionośnych.
			Ale wampiry bez krwi wpadają w szał.
			Jeśli przeprowadzamy diabolizm, przejmujemy czyjąś krew, jedną z jego mocy, zwiększamy liczniki Siły i Życia o 1.
			Diabolizm to jak zniszczenie czyjejś duszy i to najgorszy czyn przeciwko innemu wampirowi.
			
		\subsubsection{Testy}
			\begin{itemize}[noitemsep]
				\item \larpcardlinepassive{$X$} - Moc pasywna, działa ciągle. Wróg musi rzucić więcej, lub równą ilość sukcesów co $X$, aby przezwyciężyć moc. Używa monet w ilości cechy \textbf{Siła}. Użytkownik mocy nie rzuca nic.
				\item \larpcardlineversus{$X$}{$Y$} - Atak, obie strony rzucają. Atakujący rzuca $X$ monetami, a broniący się tyloma, ile wynosi jego cecha $Y$. Atakujący musi wyrzucić równią ilość lub więcej, aby zdać. Każdy sukces nadwyżki oznacza zadane obrażenia broniącemu się. Jeśli atakowany wygrywa, unika i nikt nie otrzymuje obrażeń.
				\item \larpcardlinedifficulty{$X$}{$Y$} - Test z określonym poziomem trudności. Rzucasz $X$ monetami i musisz mieć przynajmniej $Y$ sukcesów.
				\item Automatyczni wrogowie mają ustaloną z góry ilość sukcesów w obronie i ataku, które trzeba przezwyciężyć.
				\item Aby kogoś wyssać, musimy go pochwycić z \textbf{Siła}. Ssanie nie wymaga testu i zabiera akcję. Po wypiciu jednej jednostki druga osoba może się wyszarpywać z \textbf{Siła}.
				\item Pomoc komuś w akcji dodaje punkt cechy. Atak w kilku wrogów wymaga podzielenia puli cechy na części.
			\end{itemize}
			
		\subsection{Mapa}
			Zamek składa się z pokoi i korytarzy.
			Przechodząc przez korytarz musimy odkryć karteczkę i zobaczyć, co się pod nią skrywa.
			Często jest walka albo testy na uniki.
			Przedmioty możemy brać.
			
			Aby odgruzować przejście, należy przesypać łyżeczką ryż.
	
\section{Mapa}
	Najlepiej jeśli mapa przypomina graf z niewielką ilością cykli.
	Niektóre ścieżki mogą być jednokierunkowe.
	Również należy stworzyć i oznaczyć ścieżki jako tajne przejścia, przez które tylko niektórzy mogą się przedostawać.
	
	\subsection{Kniaź}
		Zamek jest wielkim żywym Kniaziem, stworzonym dawno temu przez potężnego wojewodę Tzimisce.
		Nie wszyscy gracze o tym wiedzą.
		Potrafi atakować graczy snujących się po nim.
		Posiada rdzeń w tajnym miejscu, który steruje wszystkim.
		
		Zamek nie atakuje kapitana, chyba że przypadkiem, ponieważ używa go jako przynęty.
		Kapitan został tymczasowo porwany na początku rozgrywki.
		
		Aby odkryć rdzeń, należy znaleźć jego 3 różne współrzędne \ref{sec:wspolrzedne}.
		Po znalezieniu 3 współrzędnych, gracze dowiadują się, że rdzeń się porusza i należy znaleźć jeszcze jedną wpółrzędną.
		Po znalezieniu 4 współrzędnych okazuje się, że rdzeń jest wtopiony w ciało kapitana, który o tym nie wie.
		Musiało się to stać, gdy został tymczasowo porwany.
		Jeśli kapitan nie żyje, rdzeń będzie kolejno w Szlachcicu Alfa, Potworze lub jakimś nowo-powstałym bojowym Szlachcicu.
		Rdzeń może być atakowany, ale ma \textbf{Siła} 10, więc bardzo trudno go zniszczyć.
		Platynowe ostrze potrafi natychmiastowo zniszczyć rdzeń.
		
	\subsection{Przejścia}
		Zamek atakuje graczy.
		W kilku miejscach są poustawiane kupki z kartami \ref{sec:zamek}.
		Gracze przechodząc obok kupki biorą kartę z wierzchu i sprawdzają, co się dzieje.
		Kartę bierze zawsze jeden gracz, który pierwszy dostaje lub atakuje.
		Zamek atakuje gracza, który ostatnio go atakował.
		Gracze mogą uciec, wtedy zostawiają kartę na wierzchu.
		Po pokonaniu/ominięciu wroga karta ląduje na spodzie stosu.
		
		Zakon Św. Leopolda był świadomy prawdy o zamku, więc postanowił wykorzystać sytuację, aby pozbyć się problemu raz na zawsze.
		Wykorzystują moździerze aby zawalać zamek.
		Co jakiś losowy czas Mistrzowie Larpu biorą kolejny punkt z listy przejść do zawalenia i ustawiają je jako zawalone, a jeśli już było, to zamek się samoczynnie leczy i przejście przestaje być zawalone.
		Gracze muszą spędzić czas na odgruzowywaniu, jeśli chcą przejść dalej.
		Aby ustawić zawał, należy zablokować przejście na mapie i napisać karteczkę o zawale.
		Obok zostawić kubeczki z ryżem, albo inne zadania, które gracze muszą rozwiązać, aby odgruzować przejście.
		
		Jeśli w czasie walki graczy przyjdzie zawał, wszyscy walczący dostają 2 ciężkie obrażenia.

	\subsection{Szlachcice}
		Wśród zakamarków i tajnych przejść czasami snują się Szlachcice, które niegdyś dbały o zamek.
		Mieszkańcy zamku są ich nieświadomi, gdyż zawsze siedzą w zakamarkach i kryją się przed wzrokiem.
		Wampiry mogą się ich napić, aby odzyskać punkty krwi.
		
		W tajnych przejściach są wyłożone karteczki z listami szlachciców.
		Gracze odkrywają karteczkę przy przejściu przez korytarz i sprawdzają na co się natknęli.
		Aby wykorzystać szlachcica do własnych celów, można użyć wampirzych mocy.
		Mogą także zostać zaatakowanym przez Zamek.
		
		Karty szlachciców, które można spotkać znajdują się w \ref{sec:szlachta}.
		Należy je połączyć z kartami Zamku \ref{sec:zamek}.
		
\section{Walka}
	Sposób opisu rozwiązywania konfliktów siłowych.
	
	Każda postać przy konflikcie dysponuje pulą punktów zapisanych przy używanej mocy lub w parametrze \textbf{Siła}, jeśli używa niestandardowych zagrywek.
	Obrona przeciwko czemuś jest zazwyczaj z parametru \textbf{Siła}, \textbf{Siła Woli}, bądź bardzo rzadko z pasywnej mocy.
	
	Punkty Siły Woli są brane z ilości pustych kratek w liczniku.
	
	Obie strony rzucają w kubku, bądź w dłoniach liczbę monet w ilości cechy.
	Różnica w liczbie zwycięstw na korzyść atakującego oznacza liczbę zadanych obrażeń, do której dodaje się siłę broni.
	Zwycięstwa na korzyść atakowanego oznaczają, że udało się jej uniknąć konsekwencji, chyba że posiada odpowiednią pasywną moc do kontrataku.
	
	Jeśli ktoś chce kogoś zaatakować wręcz, to musi takie osoby wpierw pochwycić lub zagonić w kozi róg, chyba że jest niewidzialny, bądź złapie je fizycznie z zaskoczenia.
	Atak dystansowy nie wymaga pochwycenia, a atakowana osoba jest informowana o tym, że atakujący w nią strzela.
	Jeśli jest kilka osób, które chcę walczyć ze sobą, to rzucają do skutku z parametru \textbf{Siła} i kto będzie miał najwięcej zwycięstw, zaczyna atak.
	Osoba atakująca ma dodatkową monetę, jeśli ktoś jej pomaga w ataku.
	Jeśli celem ataku jest kilka osób, wtedy należy podzielić swoją pulę punktów i zaatakować każdego osobno jej częścią.
	
	Atakujący zadaje obrażenia, osoba atakowana prawie nigdy nie zadaje obrażeń.
	
	Po wykonaniu akcji, atakowana osoba może próbować uciekać bądź atakować agresora z powrotem.

\section{Narzędzia}
	Lista narzędzi, które warto mieć:
	\begin{enumerate}
		\item Sznurki i smycze do zawieszenia kart na graczach.
		\item 5 monet krwi dla każdego gracza czerwono-czarnych -- jako licznik krwi i do przeprowadzania testów wzbudzenia.
		\item 10 monet cechy dla każdego gracza -- do przeprowadzania testów.
		\item Licznik Życia -- zestaw karteczek lub kratek na karcie do zamalowywania w stanach pełnego życia, lekkiego obrażenia i ciężkiego obrażenia.
		\item Licznik Siły Woli -- tak samo jak Życie.
		\item Czapki -- do oznaczania że użytkownik jest niewidzialny. Mogą być czapeczki z papieru z symbolem przekreślonego oka.
		\item Zestaw karteczek i pisadło -- dla syna Księcia, aby mógł przekazywać prywatnie podświadome rozkazy.
		\item Krótki kijek na sztylet -- dla syna Księcia.
		\item Czarna taśma -- dla wyklętego syna, do oznaczania uszkodzonych przedmiotów i części ciała.
		\item Pęk kluczy lub gąbki na smyczy -- dla Klucznika, jako broń i element fabularny.
		\item Kawałki rurek do zakładania na ręce -- dla Kapitana, aby odgrywać transformacje dłoni w szpony.
		\item Hełm, kask lub maska z zakrywaną twarzą -- dla gracza Szeryfa, aby odgrywać zastraszanie twarzą.
		\item Fiolka z cieczą lub zielona taśma klejąca -- dla gracza Zabójczyni, aby odgrywać polanie czegoś kwasem lub trucizną.
		\item Karty do budowania domków z kart -- dla gracza Duczeskiego, w liczbie 35.
		\item Przenośne stojaki -- dla gracza Duczeskiego, do odgrywania stawiania automatonów.
		\item Mały pistolecik na krew -- dla gracza Duczeskiego, do odgrywania pistoletu na krew. Może być także pistolet na wodę.
		\item Pistolet na wodę -- dla gracza Siostry Zakonnej.
		\item Srebrne/ogniste bronie przeciwko wampirom w formie długich pianek. Dla gracza Ojca zakrzywiona pianka.
		\item Lina z pętlą, prześcieradło -- dla gracza Ducha. Zakładając na kogoś pętlę, symbolizuje jego opętanie.
		\item Kubeczki przywiązane do deski, ryż i łyżeczka -- mechanika odgruzowywania, można zastąpić czymś, co wstrzyma graczy na pewien czas.
		\item Kolorowe materiały do oznaczania przejść -- do normalnych przejść oraz do tajnych przejść.
		\item Zestaw przylepnych karteczek i pisadło -- dla Szlachcica Alfa, do przyklejania na siebie i zmieniania wyglądu.
		\item Święte granaty ręczne -- dla gracza Ojca
		\item Świeczka i zapałki, ewentualnie zapalniczka -- dla gracza Siostry, aby święcić wodę. i wzniecać pożary.
		\item Fajne źródło wody i zbiorniki -- dla gracza Siostry, do święcenia. Dla reszty do obmycia się z kwasu.
		\item Mapka z tajnymi przejściami -- dla Klucznika, Ducha, Szlachcica Alfa.
		\item Lista kolejnych zawałów -- dla Mistrzów Gry oraz gracza Ojca. Numery przejść, jakie będą kolejno zawalane przez Inkwizycję.
		\item Żelki w kształcie robali -- dla Obrońcy Włości, aby odgrywać składanie jaj w graczach.
		\item Czerwona peleryna -- dla Zabójczyni, jeśli w grze jest Obrońca Włości. Aby odgrywać pożarcie Czerwonego Kapturka.
		\item Zestaw książek -- na mechanikę zdobywania informacji, do książek należy wsadzić skrawki, współrzędne i trochę krwi.
		\item Prawdziwy miecz -- na mechanikę Platynowego Ostrza, które może zniszczyć rdzeń. Ułożyć go obok najciemniejszego miejsca w zamku.
		\item Przepychaczki -- dla Ogrodnika, do odgrywania kołków osikowych. Jeśli nie można mieć więcej przepychaczek, to przylepne karteczki z opisem, że wbity jest kołek.
		\item Łańcuch lub lina z pętelką -- dla Ogrodnika, jako broń.
		\item Poduszka -- dla Ogrodnika, do odgrywania zamiany w Shreka poprzez wsadzenie jej sobie pod koszulkę.
		\item Licznik kuchenny lub aplikacja w komórce -- dla Ludwika, do odgrywania uruchamiania różnych urządzeń.
		\item Lampa światła ultrafioletowego -- dla Ludwika, do odgrywania broni lampy.
		\item Wielka strzykawka -- dla Ludwika, do odgrywania pobierania krwi.
		\item Urządzenie piszczące -- dla Ludwika, do odgrywania odstraszacza komarów. Może być komórka z generatorem sygnałów.
		\item Kamerka -- dla Ludwika, do odgrywania kamerki. Może być komórka z kamerką.
		\item Karty pożarów \ref{sec:fires} i pisadło -- dla wszystkich potrafiących wzniecać pożary.
		\item Duża łyżka --  dla gracza Duczeskiego, do szybszego przesypywania ryżu, jeśli wspomaga się swoim automatonem.
	\end{enumerate}
	
\section{Mechanika}
	Lista mechanik, które przestrzegają dane postacie.
	
	Mechaniki wspólne dla wszystkich:
	\begin{itemize}
		\item Test cechy -- rzuć monetami w liczbie cechy, musisz wyrzucić liczbę sukcesów $\geqslant$ jak wróg.
		\item Obrażenia otrzymuje atakowana strona w liczbie różnic sukcesów.
		\item Liczniki Siły Woli i Życia są w formie trójwartościowych wskaźników. Brak obrażeń, obrażenie lekkie i ciężkie.
		\item Możesz przerzucić zwykłą monetę za powierzchowne obrażenie Siły Woli.
		\item Test Siły Woli -- rzuć monetami w liczbie pustych kratek Siły Woli.
		\item Opętane osoby mogą próbować się uwolnić gdy tylko otrzymają obrażenia.
	\end{itemize}

	Mechaniki dla użytkowników krwi:
	\begin{itemize}
		\item Ilość monet krwi oznacza poziom krwi.
		\item Pobudzenie krwi polega na rzucie jedną monetą krwi i stracie jej przy czarnym wyniku.
		\item Pobudź krew, aby wyleczyć 1 powierzchownych obrażeń.
		\item Pobudź krew, aby wzmocnić się na jeden test o 1 punktów siły, można powtarzać.
		\item Przy użyciu każdej dyscypliny pobudź krew.
		\item Oddaj jeden punkt krwi wampirowi w letargu od obrażeń, aby uleczyć mu jedno ciężkie obrażenie i natychmiastowo obudzić. Oddaj więcej krwi, aby nie wpadł od razu w szał.
	\end{itemize}

	Mechaniki dla wampirów:
	\begin{itemize}
		\item Po utracie życia zapadasz w letarg i ktoś musi przekazać ci krew, abyś wstał.
		\item Po wbiciu kołka w serce, tracisz całą krew i wpadasz w letarg. Po usunięciu kołka budzisz się i wpadasz w szał.
		\item Po wyjściu z letargu pobudzasz krew, co może spowodować wejście w szał.
		\item Po utracie krwi twoja Bestia wpada w szał.
		\item Po utracie Siły Woli załamujesz się i wpadasz w depresję. Wywołuje się twoja kompulsja klanowa.
		\item Gdy twoja Bestia wpada w szał, to chce kogoś natychmiast wyssać. Twoja siła jest podwojona na ten czas.
		\item Srebrna broń zadaje ci obrażenia poważne, a normalna powierzchowne.
		\item Aby zdiabolizować kogoś, musisz wyssać mu wszystkie jednostki naczyń krwionośnych.
		\item Po utracie jednostki naczynia krwionośnego, twoja maksymalna ilość krwi maleje o 1.
		\item Po zdiabolizowaniu kogoś otrzymujesz jego specjalną moc. Zwiększasz swoją cechę \textbf{Siła} i Życie o 1.
		\item Aby kogoś wyssać, musisz go pochwycić ze swojej cechy \textbf{Siła} i wypić jedną jednostkę. Pochwycona osoba może się wyszarpywać, także robiąc testy. Po próbie wyszarpania się możesz dalej ssać bez testów, albo wykonać coś innego.
		\item Woda święcona zadaje ci lekkie obrażenia i zmniejsza o 1 cechę \textbf{Siła} aż nie wyjdziesz z pokoju.
		\item Jeśli napijesz się wody święconej, tracisz 1 punkt krwi.
	\end{itemize}

	Mechaniki dla śmiertelników:
	\begin{itemize}
		\item Po utracie życia jesteś w stanie krytycznym na granicy śmierci.
		\item Po utracie Siły Woli wpadasz w depresję.
		\item Bronie zadają ci poważne obrażenia.
		\item Masz punkty krwi, ale nie używasz ich do niczego. Po utracie wszystkich wpadasz w stan krytyczny.
	\end{itemize}

	Mechaniki dla ghuli:
	\begin{itemize}
		\item Po utracie krwi nie wpadasz w szał.
		\item Odzyskujesz punkt krwi co 10 minut.
	\end{itemize}
	
	Mechaniki dla duchów:
	\begin{itemize}
		\item Jeśli ktoś cię pokropi wodą święconą, zadaje poważne obrażenia życia. Tracisz 1 punkt ektoplazmy.
		\item Jeśli ktoś pokropi wodą święconą osobę, którą opętałeś, tracisz nad nim kontrolę.
	\end{itemize}
		
\section{Krótki opis}
	To jest krótki opis zasad, jakie należy poruszyć graczom.
	\begin{enumerate}
		\item Wydrukowany opis fabularny \ref{sec:fabularny}.
		\item Czym są wampiry. Odporni na czosnek. Piją krew.
		\item Zasada Maskarady.
		\item Spokrewnienie i dziedziczenie klanów. Opisać ogólnie.
		\item Społeczeństwo. Księża i domeny. Sekty. Fabuła.
		\item Inkwizycja. Tajne służby.
		\item Fabuła. Dziwne zachowanie Księcia. Pożar. Brak wyjścia. Bombardowanie.
		\item Dyscypliny wampirów. Opisać ogólnie.
		\item Karta postaci. Organizacje. Opis. Moce. Cele.
		\item Punkty krwi. Mechanika pobudzenia. Szał. Przykład rzutu.
		\item Krew u śmiertelników.
		\item Cecha siły. Siła Woli. Życie. Obrażenia.
		\item Przeprowadzenie zwykłego testu monetami. Przykład rzutu.
		\item Test mocy dyscypliny.
		\item Wbicie kołka. Letarg. Przekazywanie swojej krwi.
		\item Przerzut.
		\item Atak z pomocnikami.
		\item Atak na wielu wrogów.
		\item Leczenie krwią. Wzmacnianie krwią. Obudzenie z letargu.
		\item Wysysanie krwi. Diabolizm. Cecha, życie i moc.
		\item Ghule.
		\item Chodzenie po zamku i karteczki. Atak wrogów w tego, kto atakował. Zawały.
		\item Ominięcie wroga. Pozostawienie karteczki dla reszty. Karteczki bez życia.
		\item Tajne przejścia tylko dla wybranych.
		\item Opętanie przez ducha i wyrwanie się na obrażenia.
		\item Woda święcona. Napicie się wody święconej. Opętany odzyskuje władzę.
		\item Zakazane księgi grozy.
	\end{enumerate}

	%TODO
		
\section{Karty}
	Wydrukować i pociąć na kawałki. Kolejne strony czasami trzeba drukować wielokrotnie.
	\newpage
	\thispagestyle{empty}
	\subsection{Skrawki}
		Pociąć i wsadzić do książek w różnych miejscach.
		\begin{multicols}{3}
			\larpcardinfo{Zamek jest wielkim Kniaziem.}
			\larpcardinfo{Kniaź to amalgamat ciał ludzkich i mocy transformacyjnych Tzimisce.}
			\larpcardinfo{Szlachcice to podwładni przerobieni na narzędzia za pomocą mocy transformacji.}
			\larpcardinfo{Nigdy nie ufaj temu, co opowiadają Malkawianie.}
			\larpcardinfo{Książę ma tajemniczy sekret.}
			\larpcardinfo{Jeden z synów Księcia pała się Oblivionem.}
			\larpcardinfo{Oblivion to tajemne moce nekromancji.}
			\larpcardinfo{Nie tylko Tremere znają Magię Krwi.}
			\larpcardinfo{Uważaj na wodę święconą.}
			\larpcardinfo{Nie próbuj wypijać agentów z Watykanu.}
			\larpcardinfo{Wieża z Kości Słoniowej przygląda się wam wszystkim uważnie.}
			\larpcardinfo{Baron ostrzy sobie zęby na tę domenę.}
			\larpcardinfo{Jak zrobisz, żeby Ventrue wypił krew z worka, to będzie fajnie rzygał.}
			\larpcardinfo{Uważaj na cenne skarby, stojące pośrodku przejść.}
			\larpcardinfo{Jeden z was może nie być tym, za kogo się podaje.}
			\larpcardinfo{Obrońca Włości lubi łaskotki po brzuszku.}
			\larpcardinfo{Osoby pokryte kwasem otrzymują podwójne obrażenia.}
			\larpcardinfo{Broń pokryta kwasem zadaje podwójne obrażenia.}
			\larpcardinfo{Ktoś z was popełnił kiedyś diabolizm.}
			\larpcardinfo{Zdiabolizowanie kogokolwiek potrafi dać ci jakieś jego moce.}
			\larpcardinfo{Ducha także da się zdiabolizować.}
			\larpcardinfo{Wysysanie upiorów nie daje krwi.}
			\larpcardinfo{Ktoś z was ma tajemniczy sekret, o którym nie wie.}
			\larpcardinfo{Naturalne ghule odzyskują krew samoczynnie.}
			\larpcardinfo{Uważaj na Święte Granaty Ręczne.}
			\larpcardinfo{W najgłębszej części zamku czeka coś niezwykłego.}
			\larpcardinfo{Ktoś tu może pożerać innych w całości.}
			\larpcardinfo{Platynowe ostrze zagłady może go zniszczyć.}
			\larpcardinfo{Zamek ma rdzeń w nietypowym miejscu.}
			\larpcardinfo{Jeśli polejesz go krwią, to możesz przejąć kontrolę.}
			\larpcardinfo{Platynowe ostrze jest zaraz obok najciemniejszego miejsca w zamku.}
			\larpcardinfo{JI 100\%.}
			\larpcardinfo{Camarilla jest także zwana Wieżą z Kości Słoniowej.}
			\larpcardinfo{Siostra Piromanka.}
			\larpcardinfo{Woda zmywa kwas. Byle nie święcona.}
			\larpcardinfo{Pocałuj Pannę Młodą.}
			\larpcardinfo{Książę został połknięty przez zamek.}
			\larpcardinfo{Kto zostanie połknięty, staje się jego częścią.}
		\end{multicols}

	\newpage
	\thispagestyle{empty}
	\subsection{Szlachta}
	\label{sec:szlachta}
		\begin{multicols}{2}
			\larpcardenemy{Szlachcic sprzątający}{
				Staje przed wami kreatura w zjedzonym przez mole stroju pokojówki.
				Zamiast jednej dłoni ma szufelkę z własnych kości, a zamiast drugiej szczoteczkę z kręconych ludzkich włosów.
				Przepycha kurz z miejsca na miejsce i rozmazuje pajęczyny.
				}{
				}{3}{4}{1}
			\larpcardenemy{Szlachcic deratyzacyjny}{
				Kreatura brudna i ufajdana błotem, z której zwisają szkielety licznych zwierzątek.
				Patrzy się na was złowieszczo, abyście dołączyli do jego kolekcji.
				}{
				\larpcardlineattack{Z torsu natychmiastowo wyskakuje mu pazurzasta łapa, używana normalnie do łapania szczurów, atakuje tylko raz na początku.}{3}
				}{2}{3}{2}
			\larpcardenemy{Szlachcic towarowy}{
				Wsuwa się przed was kreatura niska i szeroka. Posiada 2 dodatkowe nogi i szerokie stopy.
				Potężne mięśnie na ramionach i olbrzymie dłonie.
				Łazi niezgrabnie z miejsca na miejsce.
				}{
				}{4}{5}{3}
			\larpcardenemy{Szlachcic obronny}{
				Wielkie bydlę z 4 rękoma, a każda zamiast dłoni ma srebrny miecz przytwierdzony do kości.
				Już po was.}
				{
				\larpcardlineattack{Zadaje poważne obrażenia i przecina na pół. Zawsze atakuje 4 razy.}{3}
				}{7}{2}{5}
			\larpcardenemy{Szlachcic obliczeniowy}{
				Dziwna struktura na dwóch nogach, z której otworów kapie atrament i wylatują strzępki papieru.
				Patrzy się na was i coś notuje.}
				{
				\larpcardlineattack{Oblicza jedną współrzędną. Nie zrobi tego dobrowolnie.}{-}
				}
				{6}{7}{1}
			\larpcardenemy{Szlachcic zachęcający}{
				Chudy pan w nienagannym stroju. Niby z wyglądu człowiek. Z wyjątkiem pustych oczodołów, przez które można zajrzeć do pustej czaszki.
				Odzywa się piękną polszczyzną i zaprasza do wejścia w udekorowaną dziurę w ścianie.
				}
				{
				\larpcardlineattack{Kto dobrowolnie skorzysta z zaproszenia szlachcica (serio?), zostaje elegancko zaabsorbowany w mury zamku i kończy zabawę.}{-}
				}
				{2}{1}{3}
			\larpcardenemy{Szlachcic popsuty}{
				Coś nie poszło przy generacji tego szlachcica. Wygląda jak szkielet, do którego niedbale podoklejano kawałki mięsa.
				Prosi was o skrócenie jego cierpień.
				}
				{
				}
				{3}{9}{1}
			\larpcardenemy{Szlachcic matecznikowy}{
				Wychudzona kobieta, niczym anorektyczka, z wielkim brzuchem chwieje się na nogach. Z macicy cieknie jej obleśny płyn.
				}
				{
				\larpcardlineattack{Szlachcic napina się, a z jej macicy strzela w ciebie jej płód i przykleja się do twarzy. Próbuje wygryźć nos, usta i oczy.}{4}
				
				\larpcardlineattack{Płód zadaje ci lekkie obrażenia zarówno Życia, jak i Siły Woli. Gdy się obronisz, odrywasz go od siebie i ciskasz o ścianę.}{4}
				}
				{4}{3}{5}
			\larpcardenemy{Szlachcic ogrodniczy}{
				Ogrodnik w poplamionych prochem ogrodniczkach, z wiadrem wijących się ludzkich palców.
				Posiada grabie umazane w ziemi grobowej.
				}
				{
				\larpcardlineattack{Zagrabienie. Zadaje lekkie obrażenia.}{3}
				}
				{4}{6}{3}
		\end{multicols}
	
	\newpage
	\thispagestyle{empty}
	\subsection{Zamek}
	\label{sec:zamek}
		\begin{multicols}{2}
			\larpcardenemy{Ręce wychodzące z murów}{
				Ze ścian wychodzą ręce i próbują złapać wszystko, co się rusza.
				}{
				\larpcardlineattack{Każdy, kto próbuje przekroczyć obszar, jest łapany i rozrywany.}{4}
				}{-}{3}{1}
			\larpcardenemy{Paszcza w podłodze}{
				W podłodze otwiera się potężna paszcza z zębami.
				Nie można jej atakować, a jedynie przeskoczyć.
				}{
				}{5}{-}{-}
			\larpcardenemy{Kolce ze ścian}{
				Długie kolce wysuwają się i strzelają w kogokolwiek, kto spróbuje przejść naprzeciwko.
				Nie można ich atakować.
				Można próbować prześlizgnąć się obok.
				}{
				\larpcardlineattack{Jeśli nie udało się prześlizgnąć, kolec wyskakuje z impetem i robi szaszłyk.}{6}
				}{4}{-}{-}
			\textrepeat{5}{
				\larpcarditem{Krew w worku}{
					Uzupełnia jedną jednostkę krwi.}
					}
			\textrepeat{5}{
				\larpcarditem{Dziwna krew w worku}{
					Krew ma 50\% szansy na uzupełnienie jednostki krwi. Rzuć monetą.}
					}
			\larpcardenemy{Miażdżące ściany}{
				Ściany chcą przerobić na placek każdego, kto próbuje zwyczajnie przebiec.
				}{
				\larpcardlineattack{Jeśli nie udało się przebiec, sprawdź, ile obrażeń zadały.}{9}}
				{5}{-}{-}
			\larpcardenemy{Rycerz}{
				Ze ściany wychodzi rycerz w skórzanej zbroi. Jest nadal podłączony macką do ściany.
				Macka ma dwukrotnie mniejszą siłę.
				}{
				\larpcardlineattack{Szturch. Rycerz wali mieczem.}{5}
				}{6}{4}{3}
			\larpcardenemy{Skarb}{
				Na środku przejścia na stojaczku pojawia się najcenniejszy dla ciebie skarb.
				Nie możesz się mu oprzeć. Każdy musi wykonać test Siły Woli.
				}{
				\larpcardlineattack{Mam cię! Stojaczek chowa się w podłodze, wśród setek straszliwych zębów. Ty chowasz się razem z nim.}{10}}
				{4}{-}{-}
			\larpcardenemy{Wybucha pożar}{
				Przejście staje w szalejących płomieniach. Należy jakoś ugasić pożar. Spopieli każdego, martwego lub nie, który spróbuje przebiec.
				}{
				\larpcardlineattack{Postrach wampirów. Każdy wampir stojący naprzeciwko musi wykonać test Siły Woli, czy nie ucieknie w popłochu.}{4}
				}
				{3}{4}{-}
			\larpcardenemy{Basen kwasu}{
				W złowrogozielonym basenie pływają niestrawione szczątki dziwacznych istot.
				Można próbować przeskoczyć po trupach. Komu się nie uda, zostaje pokryty kwasem.
				Otrzymuje podwójne obrażenia, ale jego broń także zadaje podwójne obrażenia.}
				{
				}
				{4}{-}{-}
			\larpcardenemy{Grząski teren}{
				Błoto ze sproszkowanych kości, zlepionych krwią zagradza drogę. Można próbować przebiec, albo przepłynąć czymś.}
				{
				\larpcardlineattack{Bul bul. Wampiry nie oddychają ale inne nieprzyjemne rzeczy czekają cię w tej brei.}{5}
				}
				{6}{-}{1}
			\larpcardenemy{Komary}{
				Z dziury wylatują, na każdego po jednym, komary wielkości psa. W kolorze skóry. Z mnóstwem ludzkich oczu.
				Syk wciąganego powietrza zagłusza trzepot skrzydeł ze skóry.
				}
				{
				\larpcardlineattack{Ssawka przysywa się do ciebie wypija ci jednostkę krwi. Razem z mięśniami i ścięgnami.}{6}
				}
				{3}{1}{1}
			\larpcardenemy{Płytki spadliwe}{
				Płytki na podłodze nie wyglądają zbyt dobrze.
				Jeśli nie zamierzasz przelecieć, albo superszybko przebiec, to musisz wykonać 5 kolejnych testów rzutu monetą, jeśli któryś się nie uda, znaczy że źle stanąłeś i spadasz w dół.
				Inni nie muszą wykonywać $i$-tego testu, jeśli ty zdałeś, bo mogą iść twoją drogą.
				}
				{
				\larpcardlineattack{W dole dziury są macki i mają ochotę rekonstruować z tobą sceny z hentai. Otrzymujesz powierzchowne obrażenia Siły Woli.}{7}
				}
				{-}{-}{-}
			\larpcardenemy{Krwisty bukłak}{
				Przed wami spada z sufitu postać przypominająca bukłak z cieczą. Ma cienkie antenki, którymi wszystkich smaga.
				Masz 50\% szansy, że pijąc to coś napijesz się specjalnie spreparowanej krwi, która daje punkty, ale na koniec picia wywołuje kompulsję. Rzuć monetą.
				}
				{
				\larpcardlineattack{Zostałeś smagnięty antenką. Obrzydlistwo.}{1}
				}
				{3}{4}{8}
			\larpcardenemy{Macka}{
				Z podłogi wyrasta macka jak u ośmiornicy. Skąd ośmiornica tutaj?
				Jeśli nie uda się przejść bokiem, to łapie kogoś z was i rzuca na najbliższą osobę, oboje otrzymujecie takie same obrażenia.
				Jeśli brak jest kogoś innego, to trzyma w uścisku, zadając obrażenia, aż ktoś się nie pojawi. Trzymana osoba może próbować się wyswobodzić.
				}
				{
				\larpcardlineattack{Złapanie i rzucenie w kogoś.}{4}
				
				\larpcardlineattack{Uścisk jednej osoby i czekanie.}{5}
				}
				{5}{3}{4}
			\larpcardenemy{Żyła z krwią}{
				Smakowita żyła grubości 30 cm przecina korytarz. Spróbuj, co się może stać?
				}
				{
				\larpcardlineattack{Ktokolwiek dotknie żyły, zostaje przez nią oplątany jak boa dusiciel. Było warto.}{9}
				}
				{-}{3}{10}
			\larpcardenemy{Panna młoda}{
				Ze ściany wystaje korpus panny młodej z długimi czarnymi włosami, w postrzępionej sukni.
				Prosi, aby ją pocałować. I przy okazji coś upić.
				Możesz przejść bokiem.
				}
				{
				\larpcardlineattack{Odkrywasz welon, aby dokonać czynu, lecz jej twarz jest nieopisywalna zdrowymi dla zmysłów słowami. Widok zadaje ci obrażenia siły woli. Ale co się napiłeś, to twoje.}{3}
				
				\larpcardlineattack{,,Już mnie nie kochasz?'' Panna atakuje każdego, kto nie będzie chciał jej pocałować, albo ją zaatakuje.}{4}
				}
				{2}{1}{5}
			\larpcardenemy{Lalkarz i laleczka}{
				Z czeluści sufitu zwisa na sznurkach kukiełka. Brakuje jej oka, a drugie jest niezdrowo wykrzywione. Sznurki są dziwnie grube i chyba nie są zrobione ze sznurka. ,,Chodźcie, pobawcie się ze mną.'' Ignoruje osoby chodzące samotnie.
				}
				{
				\larpcardlineattack{Jeśli ktoś się zbliży, kukiełka omotuje go swoimi nićmi i podwiązuje pod sufitem. Gracz musi raz zaatakować któregoś z towarzyszy w dowolny sposób, a potem się zrywa.}{6}
				}
				{4}{5}{3}
			\larpcardenemy{Jęzor}{
				Ze szczeliny w ścianie wyskakuje jęzor i atakuje natychmiast pierwszą osobę.
				Jeśli się mu uda, zostaje ona pożarta, dostaje obrażenia i może się przenieść do dowolnego miejsca w zamku.
				}
				{
				\larpcardlineattack{Natychmiastowy atak pojedynczy. Jeśli się nie udało, dostajesz obrażenia i zostajesz wykupkany w dowolnej części zamku.}{6}
				}
				{4}{6}{3}
			\larpcardenemy{Korytarz nieeuklidesowy}{
				Tu, gdzie powinno być jedno przejście, są 4 różne. Tylko jedno jest poprawne.
				Każdy rzuca 2 monetami i przechodzi bez szwanku tylko jeśli ma oba sukcesy.
				Inaczej jest gryziony przez kły czasoprzestrzeni.
				}
				{
				\larpcardlineattack{Ugryzienie czegoś ponadwymiarowego. Tutaj, gdzie czas zwija się w kłąb, wszystkie zadane przez wroga obrażenia tak na prawdę cię leczą. Potem odnajdujesz drogę.}{4}
				}
				{-}{-}{-}
			\larpcardenemy{Ściana}{
				Przed wami stoi ściana, zupełnie jak prawdziwa.
				Osoby wyczulone na niewidzialność i właściciele kamer mogą stwierdzić, że nie jest prawdziwa i przez nią przejść.
				Pozostali muszą poszukać innej drogi.
				}
				{}
				{-}{-}{-}
		\end{multicols}
		
	\newpage
	\thispagestyle{empty}
	\subsection{Pożary}
	\label{sec:fires}
		\begin{multicols}{3}
			\textrepeat{10}{
				\larpcardenemy{Wzniecony pożar}{
				To miejsce się pali, chyba ktoś wzniecił pożar.
				}{
				\larpcardlineattack{Postrach wampirów. Każdy wampir stojący naprzeciwko musi wykonać test Siły Woli, czy nie ucieknie w popłochu.}{5}
				}
				{3}{\Square}{-}
			}
		\end{multicols}

	\newpage
	\thispagestyle{empty}
	\subsection{Współrzędne}
	\label{sec:wspolrzedne}
		\begin{multicols}{3}
			\larpcarditem{Współrzędna Rdzenia}{$\varLambda \varUpsilon \varPsi$}
			\larpcarditem{Współrzędna Rdzenia}{$\varOmega \varTheta \varGamma$}
			\larpcarditem{Współrzędna Rdzenia}{$\varDelta Z \varXi$}
			\larpcarditem{Współrzędna Rdzenia}{$T \varPi \varDelta$}
			\larpcarditem{Współrzędna Rdzenia}{$A K \varPhi$}
			\larpcarditem{Współrzędna Rdzenia}{$\varGamma H X$}
			\larpcarditem{Współrzędna Rdzenia}{$GMD$}
			\larpcarditem{Współrzędna Rdzenia}{$\varPi P \varPhi$}
			\larpcarditem{Współrzędna Rdzenia}{$\varDelta \varXi \varDelta$}
		\end{multicols}

	\newpage
	\thispagestyle{empty}
	\subsection{Postaci}
	\label{sec:characters}
		Postacie są uporządkowane w kolejności, w jakiej należy je przyporządkowywać kolejnym osobom biorącym udział w zabawie.
		Minimalna ilość graczy to 4. Jest potrzebny jeden dobrze ogarnięty Mistrz Larpa, albo dwóch zwyczajnych.
		Jeśli w grze jest Szlachcic Alfa, to warto dodać jeszcze jednego Mistrza.
		Dalsze postacie ogólnie można ustawiać wedle uznania, jeśli ktoś chce grać np. kimś mniej morderczym, ale to nie jest zalecane.
		
		\thispagestyle{empty}
		\larpcardcharacter{Teodor Szymonowicki}
			{\larpcardlinespecies{Wampir Ventrue}{Dziedzic i syn Księcia}{Dworzanie Księcia, Camarilla}}{
			Twój ojciec, Książę, pałał się Nekromancją.
			Obrzydzało cię to niemiłosiernie.
			Robiłeś wszystko, aby pewnego razu jego praktyki zemściły się na nim, a ty będziesz mógł zostać jego następcą.
			Twój ojciec wydziedziczył twojego brata i nie wiesz, za co.
			Kiedyś zdiabolizowałeś Malkawianina i bardzo pragniesz to ukryć.
			Trenowałeś malkawiańskie praktyki obłędu, które w trakcie kłótni przypadkowo użyłeś na ojcu, przez co wpadł w szał i podpalił zamek.
			Jako Ventrue masz wysublimowane podniebienie i krew z worka nie daje ci punktów krwi.}
			{
				\begin{itemize}[noitemsep]
					\item \larpcardlinepower{Niewidoczność}{Zasłona cienia}{Pobudzenie, \larpcardlinepassive{5}}{Pobudzasz krew, gdy nikt nie patrzy i wtapiasz się w tło i cię nie widać. Wróg może cię wypatrzeć, jeśli zda test.}
					\item \larpcardlinepower{Dominacja}{Obłęd}{\larpcardlineversus{7}{Siła Woli}}{Nieświadomie zadajesz powierzchowne obrażenia Siły Woli.}
					\item \larpcardlinepower{Dominacja}{Mesmeryzm}{\larpcardlineversus{5}{Siła Woli}}{Rozkazujesz komuś natychmiast wykonać konkretne instrukcje.}
					\item \larpcardlinepower{Dominacja}{Podświadome instrukcje}{\larpcardlineversus{6}{Siła}}{Zapisujesz w kimś odpowiedni warunek i akcję, którą ma wykonać, w formie tekstu na karteczce: \emph{Jeśli stanie się X to zrób Y.} Druga osoba wykonuje w tajemnicy test \larpcardlinepassive{5}, czy nie udało jej się przezwyciężyć komendy.}
					\item \larpcardlinepower{Broń}{Sztylet}{\larpcardlineweapon{+0}}{Atak z bliska.}
				\end{itemize}

			}{
				\begin{enumerate}[noitemsep]
					\item Zwalić winę za obłęd Księcia na kogoś innego.
					\item Zniszczyć każdego, kto próbuje używać Nekromancji.
					\item Nie pozwolić, aby ktokolwiek dowiedział się o twoim diabolizmie.
					\item Przeżyć.
					\item Zostać następcą Księcia.
				\end{enumerate}
			}
			{\larpcardlineattributes{6}{8}{6}}
			{\larpcardlinecompulsion{Arogancja, żądza władzy nad innymi.}}
			{\larpcardlinepower{Dominacja}{Mesmeryzm}{\larpcardlineversus{6}{Siła Woli}}{Rozkazujesz komuś natychmiast wykonać konkretne instrukcje.}}
			
		\thispagestyle{empty}
		\larpcardcharacter{Sławomir Szymonowicki}{
			\larpcardlinespecies{Wampir Ventrue}{Wydziedziczony syn Księcia}{Dworzanie Księcia, Camarilla}}{
			Twój ojciec wyklął ciebie, gdy zacząłeś lepiej sobie radzić w mocach Nekromancji, niż on sam.
			Możesz użyć mocy, aby zobaczyć duchy w danym miejscu.
			Nie lubisz swojego brata i nie możesz pozwolić, aby przejął władzę.
			Chcesz aby władza została w rękach Camarilli.
			Oblivion to twoja broń i nie potrzebujesz żadnej innej.
			Jako Ventrue masz wysublimowane podniebienie i krew z worka nie daje ci punktów krwi.}
			{
				\begin{itemize}[noitemsep]
					\item \larpcardlinepower{Nekromancja}{Wizja Obliviona}{Pobudzenie}{Potrafisz zobaczyć wszystkie duchy w pomieszczeniu.}
					\item \larpcardlinepower{Nekromancja}{Ręce Ahrimana}{\larpcardlineversus{5}{Siła}}{Macki wychodzą z cieni, poruszają się po powierzchniach, mogą atakować i manipulować rzeczami.}
					\item \larpcardlinepower{Nekromancja}{Dotyk Obliviona}{\larpcardlineversus{5}{Siła}}{Twój dotyk powoduje poważne obrażenia \larpcardlineweapon{+2}, uszkadzając ciało bądź dotykane przedmioty. Bronie tracą jeden punkt siły, osoby tracą władzę w członkach ciała.}
					\item \larpcardlinepower{Nekromancja}{Chłód zaświatów}{\larpcardlineweapon{+2}}{Wypluwasz powiew mrozu z zaświatów. Pozwala gasić skutecznie ogień.}
					\item \larpcardlinepower{Nekromancja}{Maczanie w Oblivionie}{Pobudzenie}{Możesz zanurzyć czyjąś broń w Oblivionie, aby zadawała takie same obrażenia również na duchy. Szkoda że nie masz broni.}
					\item \larpcardlinepower{Dominacja}{Zatarcie wspomnień}{\larpcardlineversus{6}{Siła}}{Mówisz ofierze, jakie jej znane ci wspomnienia mają być zatarte podanymi.}
					\item \larpcardlinepower{Prezencja}{Zachwyt}{\larpcardlineversus{4}{Siła}}{Powodujesz, że ofiara na krótko zmienia nastawienie do ciebie i nie chce atakować.}
				\end{itemize}
			}{
				\begin{enumerate}[noitemsep]
					\item Zemścić się na ojcu.
					\item Znaleźć prawdę co się stało.
					\item Powstrzymać brata przed przejęciem władzy.
					\item Ukrywać swoje moce przed bratem.
					\item Przeżyć.
				\end{enumerate}
			}{
			\larpcardlineattributes{5}{7}{5}}
			{\larpcardlinecompulsion{Arogancja, chęć podporządkowania sobie całego najbliższego świata niematerialnego.}}
			{\larpcardlinepower{Nekromancja}{Dotyk Obliviona}{\larpcardlineversus{6}{Siła}}{Twój dotyk powoduje poważne obrażenia \larpcardlineweapon{+3}, uszkadzając ciało bądź dotykane przedmioty. Bronie tracą dwa punkty siły, osoby tracą władzę w członkach ciała i odczuwają ból.}}
			
		\thispagestyle{empty}
		\larpcardcharacter{Gerwazy}{
			\larpcardlinespecies{Wampir Malkawian}{Klucznik zamku}{Dworzanie Księcia, Camarilla}}{
			Jesteś zaufanym klucznikiem, znasz wszystkie tajne przejścia w zamku.
			Zawsze widzisz nieistniejące rzeczy i nie odróżniasz prawdy od własnych urojeń.
			Jesteś świadomy swojej klątwy.
			Znasz wszystkie dziwactwa występujące w zamku, ale zawsze brałeś je za urojenia.
			Z tego powodu nigdy nie powiedziałeś o niczym Księciu.}
			{
				\begin{itemize}[noitemsep]
					\item \larpcardlinepower{Nadwrażliwość}{Fatamorgana}{\larpcardlineversus{4}{Siła}}{Tworzysz iluzje wpływające na zmysły wszystkich wokół.}
					\item \larpcardlinepower{Nadwrażliwość}{Odczytywanie duszy}{\larpcardlineversus{6}{Siła Woli}}{Odczytujesz gatunek osoby, jej cechy i liczniki, dowiadujesz się czy popełniła diabolizm.}
					\item \larpcardlinepower{Nadwrażliwość}{Przeczucie}{Pobudzenie}{Pozwala prywatnie podejrzeć pierwszą karteczkę na stosie.}
					\item \larpcardlinepower{Nadwrażliwość}{Wyczucie niewidocznego}{Pasywna}{Pozwala w przybliżeniu wyczuć niewidzialne rzeczy w okolicy.}
					\item \larpcardlinepower{Broń}{Pęk kluczy}{\larpcardlineweapon{+0}}{Pęk bardzo ostrych kluczy na sznurku jest dobrą bronią.}
				\end{itemize}
			}{
				\begin{enumerate}[noitemsep]
					\item Uratować Księcia.
					\item Naprawić zamek.
					\item Chronić innych poddanych.
					\item Podrapać Obrońcę Włości po brzuszku.
					\item Przeżyć.
				\end{enumerate}				
			}{
			\larpcardlineattributes{4}{6}{4}}
			{\larpcardlinecompulsion{Urojenia i wizje.}}
			{\larpcardlinepower{Nadwrażliwość}{Wzmocnione Przeczucie}{Pobudzenie}{Pozwala prywatnie podejrzeć dwie karteczki na stosie i odłożyć w dowolnej kolejności.}}
			
		\thispagestyle{empty}
		\larpcardcharacter{Kamil}{
			\larpcardlinespecies{Wampir Toreador}{Kapitan Księcia}{Dworzanie Księcia, Camarilla}}{
			Jesteś zaufanym kapitanem Księcia.
			Zostałeś wciągnięty przez zamek, ale z jakichś przyczyn nadal nieżyjesz.
			Musisz pomóc absolutnie wszystkim rozwiązać ich problemy.
			Coś dziwnego się dzieje z twoim ciałem, jeśli się skupisz, i nie masz pojęcia, co.
			Możesz modyfikować swoje ręce na ostrza lub stać się zupełnie lekki.
			Lepiej nie pokazywać tych zdolności innym.
			}{
				\begin{itemize}[noitemsep]
					\item \larpcardlinepower{Transformacja}{Waga piórka}{Pobudzenie}{Zmniejszasz swój ciężar niemal do zera.}
					\item \larpcardlinepower{Transformacja}{Pierwotna broń}{Pobudzenie}{Zmieniasz ręce w ostrza i możesz atakować nimi jak srebrem \larpcardlineweapon{+4}.}
					\item \larpcardlinepower{Nadwrażliwość}{Obeah}{\larpcardlinedifficulty{6}{2}}{Leczysz tyle powierzchownych obrażeń Siły Woli, ile masz nadwyżek punktów sukcesu.}
					\item \larpcardlinepower{Przyspieszenie}{Przebiegnięcie}{\larpcardlineversus{6}{Siła}}{Potrafisz natychmiastowo przebiec przez obszar i biegać po wodzie.}
					\item \larpcardlinepower{Broń}{Miecz}{\larpcardlineweapon{+2}}{Zwykły metalowy miecz, jakich wiele.}
				\end{itemize}
			}{
				\begin{enumerate}[noitemsep]
					\item Uratować Księcia.
					\item Dowiedzieć się prawdy o zamku.
					\item Zemścić się na tym, kto porwał Księcia.
					\item Dowiedzieć się, co to za nowe moce posiadasz.
					\item Przeżyć.
				\end{enumerate}
			}{
			\larpcardlineattributes{5}{4}{6}}
			{\larpcardlinecompulsion{Obłęd na temat jakiejś rzeczy lub kogoś, możesz mówić tylko o tym.}}
			{\larpcardlinepower{Transformacja}{Metamorfoza}{Pobudzenie}{Zmieniasz się w nietoperza. Możesz chodzić dowolnie po całym zamku i nie brać karteczek. Machaj rękami i piszcz.}}
			
		\thispagestyle{empty}
		\larpcardcharacter{Ryszard}{
			\larpcardlinespecies{Wampir Nosferatu}{Szeryf Księcia}{Dworzanie Księcia, Camarilla}}{
			Całe nieżycie byłeś razem z Księciem.
			Zwykle nosisz zakryty hełm aby wzbudzać mniejszy strach wśród wrogów waszej domeny.
			Nie pamiętasz, gdy ostatni raz widziałeś swoją twarz w lustrze, a nadal masz koszmary.
			Posiadasz zastraszanie oraz znasz się na anatomii i torturach.
			Umiesz obsługiwać zaawansowane urządzenia technologiczne.
			Aby użyć obliczacza, wlewasz w niego jednostkę Vitæ i czekach 20 minut. Musisz najpierw zdobyć to urządzenie i wiedzieć, do czego służy.}
			{
				\begin{itemize}[noitemsep]
					\item \larpcardlinepower{Nosferatu}{Wygląd Nosferatu}{\larpcardlineversus{6}{Siła Woli}}{Pokazujesz swój upiorny wygląd i zadajesz powierzchowne obrażenia Siły Woli \larpcardlineweapon{+1}.}
					\item \larpcardlinepower{Animalizm}{Poskromienie Bestii}{\larpcardlineversus{6}{Siła}}{Powoduje apatię lub blokuje używanie mocy krwi.}
					\item \larpcardlinepower{Potencja}{Daleki skok}{Pobudzenie}{Pozwala przeskoczyć kawałek podłoża.}
					\item \larpcardlinepower{Niewidoczność}{Niewidoczne przejście}{Pobudzenie}{Pozwala na przemykanie, będąc niezauważalnym, jeśli się jest cicho i zniknęło się niezauważonym.}
					\item \larpcardlinepower{Broń}{Wyrąbisty miecz}{\larpcardlineweapon{+3}}{Ten miecz spopielił już wielu wrogów Camarilli, a nadal się nie stępił. Zadaje ciężkie obrażenia.}
				\end{itemize}

			}{
				\begin{enumerate}[noitemsep]
					\item Uratować Księcia.
					\item Ochronić poddanych.
					\item Zniszczyć każdego, kto popełnia diabolizm.
					\item Ochronić mury zamku, abyście dalej mogli w nim mieszkać.
					\item Zdobyć obliczacz i dowiedzieć się, co robi.
					\item Poświęcić się dla ratunku innych.
				\end{enumerate}
			}{
			\larpcardlineattributes{6}{6}{7}}
			{\larpcardlinecompulsion{Kryptofilia. Chęć poznania wszystkich możliwych sekretów wszystkiego wokół.}}
			{\larpcardlinepower{Animalizm}{Poskromienie Bestii}{\larpcardlineversus{7}{Siła}}{Powoduje apatię lub blokuje używanie mocy krwi.}}
			
		\thispagestyle{empty}
		\larpcardcharacter{Amelia}{
			\larpcardlinespecies{Wampir Banu Haquim}{Tajna zabójczyni}{-}}{
			Informacja o tym, kto jest twoim zleceniodawcą, została ci wymazana z głowy.
			Wiesz że twój klan nigdy się nie poddaje, a jeśli ktoś ma być zabity, to będzie to nieuniknione.
			Twoim celem jest szeryf.
			I warto by zwalić winę na kogoś innego.
			Posiadasz zestaw broni na wampiry.
			Uważasz, że inne wampiry nie mają prawa należeć do żadnych sekt jak Camarilla, a tylko przynależność klanowa się liczy.
			}{
				\begin{itemize}[noitemsep]
					\item \larpcardlinepower{Magia Krwi}{Korozyjne Vitæ}{Pobudzenie}{Twoja krew produkuje silny kwas, jeden test pobudzenia na 1 dm³ materiału.}
					\item \larpcardlinepower{Magia Krwi}{Stłumienie Vitæ}{\larpcardlineversus{4}{Siła Woli}}{Usuwasz innemu użytkownikowi krwi jeden punkt krwi.}
					\item \larpcardlinepower{Magia Krwi}{Dotyk Skorpiona}{Pobudzenie}{Twoja krew produkuje silną truciznę, którą można pokryć broń dla podwójnych obrażeń.}
					\item \larpcardlinepower{Magia Krwi}{Łyk Skorpiona}{\larpcardlinepassive{6}}{Używasz Dotyku Skorpiona, aby wysysaną sobie krew zmienić w truciznę. Wysysający musi zdać test.}
					\item \larpcardlinepower{Magia Krwi}{Kradzież Vitæ}{\larpcardlineversus{6}{Siła}}{Rozrywasz komuś tętnicę na odległość, aby przesikać 1 punkt krwi wprost do ust. Rana się zasklepia.}
					\item \larpcardlinepower{Broń}{Srebrny miecz}{\larpcardlineweapon{+2}}{Specjalny miecz na potwory. Nie masz drugiego na ludzi. Zadaje poważne obrażenia wampirom.}
				\end{itemize}
			}
			{
				\begin{itemize}[noitemsep]
					\item Sproszkować szeryfa Księcia.
					\item Zwalić winę na kogoś innego.
					\item Uciec.
				\end{itemize}
			}{
			\larpcardlineattributes{7}{5}{4}}
			{\larpcardlinecompulsion{Osądzenie i zabójstwo każdego, kto się nie zgadza z twoimi przekonaniami.}}
			{\larpcardlinepower{Magia Krwi}{Kradzież Vitæ}{\larpcardlineversus{7}{Siła}}{Rozrywasz komuś tętnicę na odległość, aby przesikać do 2 punktów krwi wprost do ust. Rana się zasklepia.}}
			
		\thispagestyle{empty}
		\larpcardcharacter{Dionizy Duczeski}{
			\larpcardlinespecies{Naturalny ghul}{Znajomy Księcia}{Dworzanie Księcia, Camarilla}}{
			Ghule stworzone dawno temu przez Tzimisce po wielu pokoleniach nauczyły się tworzyć własne Witæ i rozmnażać jak ludzie.
			Rodzina Duczeskich jest szanowanym rodem inteligenckim.
			Wasze badania opierają się o tworzenie skomplikowanych mechanizmów zegarowych.
			Posiadasz 5 rozkładalnych klatek, które chronią przed zawałem, rozłożenie klatki wymaga postawienia dwupiętrowego domku z 7 kart.
			Czytałeś o mocach transformacji Tzimisce i podejrzewasz, że zamek jest wielkim Kniaziem i ma rdzeń sterujący, ale nie wiesz, gdzie.
			Możesz skonstruować blokadę na rdzeń, która uśpi zamek.
			Aby skonstruować blokadę, musisz zebrać pośród zamku 5 ciekawych przedmiotów o tematyce blokowania, a potem spróbować skonstruować blokadę, pytając się Mistrza Larpu.
			Mistrz Larpu może zaakceptować albo powiedzieć, które z przedmiotów się nie nadają.}
			{
				\begin{itemize}[noitemsep]
					\item \larpcardlinepower{Wykształcenie}{Przekonywanie}{\larpcardlinepassive{7}}{Masz dar przekonywania wampirów, aby cię nie wyssali.}
					\item \larpcardlinepower{Wyposażenie}{Nakręcana zbroja}{\larpcardlinepassive{4}}{Atakujący musi zdać dodatkowy test, aby sprawdzić, czy automatyczna zbroja nie zablokowała jego ataku całkowicie.}
					\item \larpcardlinepower{Odporność}{Twardy umysł}{\larpcardlinepassive{5}}{Twoja Siła Woli używana w obronie jest wzmocniona.}
					\item \larpcardlinepower{Wyposażenie}{Automaton ratowniczy}{\larpcardlinedifficulty{5}{3}}{Wykonaj Pobudzenie krwi, aby zasilić automaton i zdaj test, aby zgasić pożar lub użyć większej łyżki do odgruzowywania.}
					\item \larpcardlinepower{Wyposażenie}{Automaton obronny}{\larpcardlinepassive{3}}{Wykonaj pobudzenie krwi, aby zasilić automaton, który będzie chronił dowolną grupę osób. Ktokolwiek was atakuje, będzie musiał dodatkowo zdać test.}
					\item \larpcardlinepower{Broń}{Zawodny pistolecik na krew}{\larpcardlineversus{3}{Siła}}{Musisz wykonać Pobudzenie krwi, aby naładować pistolet. Zadaje \larpcardlineweapon{+0} poważnych obrażeń wampirom.}
				\end{itemize}
			}
			{
				\begin{enumerate}[noitemsep]
					\item Przeżyć.
					\item Uśpić zamek za pomocą blokady.
					\item Uratować wszystkie dobre istoty.
					\item Dowiedzieć się, co się stało z Księciem.
				\end{enumerate}
			}{
			\larpcardlineattributes{4}{8}{4}}
			{}
			{\larpcardlinepower{Odporność}{Twardość}{Pobudzenie}{Negujesz jeden punkt dowolnych obrażeń, które w ciebie wchodzą.}}
			
		\thispagestyle{empty}
		\larpcardcharacter{Ojciec Arnold}{
			\larpcardlinespecies{Człowiek}{Zakonnik Inkwizycji}{Inkwizycja}}{
			Zostałeś posłany przez Zakon Św. Leopolda, aby upewnić się, że wszystko w zamku ma być ostatecznie martwe.
			Nie podzielasz jednak wszystkich idei swoich przełożonych.
			Chcesz uratować wszystkie dobre istoty w tajemnicy przed resztą zakonu.
			Posiadasz Prawdziwą Wiarę, która odstrasza wampiry i duchy.
			Zamek jest ostrzeliwany przez Inkwizycję, masz informację o wszystkich kolejnych zawałach korytarzy.
			Wiesz, że zamek jest wielkim Kniaziem stworzony dawno temu przez wojewodę Tzimisce.
			Posiada rdzeń, który wszystko kontroluje.
			Znasz jedną ze współrzędnych rdzenia.
			Możesz wykonać egzorcyzm na kimś, aby wypędzić z niego ducha.
			Potrafisz rozpoznać moce Transformacji, jeśli ktoś ich używa.
			Posiadasz 5 Świętych Granatów Ręcznych, które zadają obrażenia wszystkim duchom i wampirom w pokoju.}
			{
				\begin{itemize}[noitemsep]
					\item \larpcardlinepower{Wyposażenie}{Kolczuga}{Pasywna}{Wszystkie ciężkie obrażenia zamieniają się na powierzchowne.}
					\item \larpcardlinepower{Prawdziwa Wiara}{Niesmaczna krew}{Pasywna}{Kto zaczyna pić twoją krew, musi za każdy punkt wykonać \larpcardlinepassive{8} z Siły Woli.}
					\item \larpcardlinepower{Prawdziwa Wiara}{Odstraszenie wampirów}{\larpcardlineversus{9}{Siła Woli}}{Zmuszasz wszystkie wampiry, aby uciekły z pokoju.}
					\item \larpcardlinepower{Prawdziwa Wiara}{Wystraszenie duchów}{\larpcardlineversus{8}{Siła Woli}}{Zadajesz obrażenia Siły Woli wszystkim duchom w pokoju, jeśli w nim są}.
					\item \larpcardlinepower{Prawdziwa Wiara}{Światło Chrystusa}{\larpcardlineversus{4}{Siła}}{Zadajesz ciężkie obrażenia fizyczne wszystkim duchom w pokoju, niezależnie czy w nim są.}
					\item \larpcardlinepower{Inkwizycja}{Egzorcyzm}{\larpcardlineversus{7}{Siła Woli}}{Wypędzasz ducha z opętanej osoby.}
					\item \larpcardlinepower{Inkwizycja}{Idź precz, szatanie}{\larpcardlinepassive{7}}{Jeśli duch cię opęta, musi wykonać dodatkowy test, inaczej puszcza cię.}
					\item \larpcardlinepower{Broń}{Pastorał}{\larpcardlineweapon{+2}}{Święcony pastorał zadaje poważne obrażenia wampirom i duchom i powierzchowne śmiertelnikom.}
					\item \larpcardlinepower{Broń}{Święty Granat Ręczny}{\larpcardlineweapon{+1}}{Granat zadaje ciężkie obrażenia wszystkim wampirom i duchom w pokoju. Wyciągnij zawleczkę, policz do 3. Nie licz do 4, ani do 2, chyba że przechodzisz przez 2, aby policzyć do 3. Potem rzuć i krzyknij \emph{Alleluja}!}
				\end{itemize}
			}
			{
				\begin{enumerate}[noitemsep]
					\item Uratować wszystkie dobre istoty, w tajemnicy przed Inkwizycją i Watykanem.
					\item Nawrócić złe istoty na dobrą drogę.
					\item Przekonać Ludwika, aby sobie poszedł lub do was dołączył.
					\item Zniszczyć zamek.
					\item Przeżyć albo umrzeć za wiarę.
				\end{enumerate}

			}{
			\larpcardlineattributes{4}{6}{3}}
			{}
			{\larpcardlinepower{Prawdziwa Wiara}{Nawrócenie}{Pasywna}{Od teraz stajesz się dobry. Już nie chcesz zabijać. Chcesz wszystkich ukochać, przytulić i uratować.}}
			
		\thispagestyle{empty}
		\larpcardcharacter{Upiór Konrad}{
			\larpcardlinespecies{Widmo}{Zjawy}{-}}{
			Jesteś duchem snującym się po zamku. Zamek jest czymś na kształt wielkiego żywego potwora. Nie wiesz, skąd i jak powstał.
			Książę pałał się potajemnie nekromancją i przyzwał cię w czasie jednego ze swoich nieudanych eksperymentów.
			Jesteś domyślnie niewidoczny dla wszystkich innych, chyba że zdecydujesz się pokazać.
			Możesz kogoś opętać, aby chodził razem z tobą i wykonywał twoje rozkazy, używasz do tego liny z pętlą.
			Opętana osoba może się próbować uwolnić, gdy tylko otrzyma jakiekolwiek obrażenia.
			Gdy kogoś opętasz, to wnikasz w jego ciało i ponosisz tyle obrażeń, co on.
			Nie możesz przenikać przez ściany, bo zamek cię atakuje. Znasz pozycje tajnych przejść.
			Poważne obrażenia są dla ciebie powierzchowne, a powierzchowne żadne.
			Nie możesz fizycznie atakować nikogo żywego, atakujesz mentalnymi cechami i zadajesz obrażenia Siły Woli.
			Jednak inni mogą cię normalnie atakować, jeśli cię widzą. Zamek cię zawsze widzi.
			Jeśli znajdziesz kogoś martwego lub w letargu, możesz wyrwać mu duszę, dzięki czemu stanie się duchem podobnym do ciebie.
			Traci swoje moce, ale nie musi chcieć ci pomagać. Możesz mu przekazać jedną ze swoich spirytualnych mocy.
			Możesz przechodzić bez problemu przez zawaliska, ale przy innych przejściach musisz pobierać karteczki.
			Nie możesz nikogo łapać, a jedynie zaganiać w kozi róg.
			Zamiast punktów krwi posiadasz punkty ektoplazmy. Zużywasz je jak krew. Gdy kogoś opętasz, możesz zabrać mu jeden punkt krwi, aby odnowić ektoplazmę.
			Możesz zostać wyssany, jeśli jesteś widoczny. Wysysający zabiera ci ektoplazmę, ale nie otrzymuje za to krwi. Może popełnić diabolizm.
			Oblanie wodą święconą zadaje ci poważne obrażenia i zabiera 1 punkt ektoplazmy.}
			{
				\begin{itemize}[noitemsep]
					\item \larpcardlinepower{Spirytualizm}{Zmiana postaci}{Pobudzenie}{Pokazujesz lub ukrywasz swoją materialną postać dla wszystkich zgromadzonych w pokoju.}
					\item \larpcardlinepower{Spirytualizm}{Buuu!}{\larpcardlineversus{7}{Siła Woli}}{Przestraszasz kogoś na śmierć, zadając mu powierzchowne obrażenia Siły Woli.}
					\item \larpcardlinepower{Spirytualizm}{Największy koszmar}{\larpcardlineversus{5}{Siła Woli}}{Pokazujesz komuś najgorsze wizje z jego życia i zadajesz poważne obrażenia Siły Woli.}
					\item \larpcardlinepower{Spirytualizm}{Kompulsja}{\larpcardlineversus{6}{Siła Woli}}{Wywołujesz natychmiastowo kompulsję na danym wampirze.}
					\item \larpcardlinepower{Spirytualizm}{Przejęcie kontroli}{\larpcardlineversus{6}{Siła Woli}}{Przejmujesz kontrolę nad kimś, zakładając mu linę z pętlą.}
					\item \larpcardlinepower{Spirytualizm}{Gadanie w myślach}{Pobudzenie}{Możesz komuś powiedzieć coś w myślach i rozmawiać prywatnie i nie musisz go łapać.}
					\item \larpcardlinepower{Spirytualizm}{Wyrwanie duszy}{\larpcardlinedifficulty{6}{3}}{Tworzysz nowego ducha i dajesz mu życia w liczbie sukcesów.}
				\end{itemize}
			}
			{
				\begin{enumerate}[noitemsep]
					\item Zemścić się na Księciu.
					\item Nie pozwolić na uratowanie Księcia.
					\item Stworzyć sobie kompanów do mieszkania razem.
					\item Pozbyć się wszystkich innych z zamku.
					\item Przestraszyć kogoś do szaleństwa.
					\item Zabić zamek i zostawić mury, albo nawiązać z nim pokój.
				\end{enumerate}
			}
			{
			\larpcardlineattributes{5}{6}{3}}
			{}
			{\larpcardlinepower{Spirytualizm}{Forma niematerialna}{Pobudzenie}{Potrafisz na chwilę zmienić postać na niematerialną. Nie pozwala to na przechodzenie przez ściany, ale możesz się komuś wyrwać albo przejść przez zawalisko.}}

		\thispagestyle{empty}
		\larpcardcharacter{Szlachcic Alfa}{
			\larpcardlinespecies{Mięsny golem, ghul}{Kreacja Zamku}{Zamek}}{
			Zamek jest potężnym Kniaziem, czyli amalgamatem pochwyconych przez dawnego właściciela Tzimisce wrogów, stopionych razem w wirującą masę i ożywionych magią.
			Kniaziem tak potężnym, że stworzył ciebie, abyś usunął z jego rejonów wszystkich żywych i nieżywych.
			Masz zaawansowane moce transformacji, które pozwalają ci przybierać wygląd normalnego wampira lub człowieka.
			Wiesz, że rdzeń, który steruje zamkiem, został wtopiony w ciało kapitana, a on o tym nie wie.
			Powinieneś go chronić tak, aby nikt się nie domyślił.
			Możesz zapytać się Mistrzów Larpa o wszystkie informacje na temat zamku jak miejsca pożarów i zawałów.
			Możesz się poruszać dowolnie po zamku z każdego miejsca w każde. Ale takie wchodzenie w ściany może być bardzo podejrzane.
			Zamek nie atakuje ciebie i nie musisz odkrywać karteczek (ale niech inni o tym nie wiedzą).
			Gdy zamek zostanie przez kogoś zraniony, możesz napić się z rany jego krwi, aby odzyskać 1 punkt krwi.
			Wiesz, że Książę został porwany i zaabsorbowany przez zamek i poszukiwania jego są daremne, ale niech próbują jak najdłużej.
			Jeśli ktoś leży martwy lub w letargu, możesz wskazać zamkowi jego pozycję, aby go zaabsorbował.
			Jeśli zwabisz kogoś do najgłębszej części zamku i będziecie tam sami razem, to zostaje automatycznie absorbowany, nawet jak jeszcze się rusza.
			Wtedy zmienia się w twojego pomocnika i może ci pomagać. Jednak nie ma ludzkiego wyglądu, traci wszystkie moce i umysł.
			Jeśli zmieniasz wygląd, zapisujesz opis na przylepnej karteczce i przyklejasz sobie, lub komuś.
			Możesz także dać kogoś dobrowolnie \emph{przerobić} na Szlachcica.
			Zaprowadzasz go do najgłębszej części zamku, gdzie odbywa się bolesna transformacja wedle twojego upodobania. Gracz zachowuje umysł, moce i cechy.
			Przekazujesz mu jedną z dodatkowych mocy, którą musisz uzgodnić z Mistrzami Larpu.
			Jego \textbf{Siła} i Życie podwajają się.}
			{
				\begin{itemize}[noitemsep]
					\item \larpcardlinepower{Zamek}{Wskazanie ofiary}{Pobudzenie}{Pobudzasz krew, aby przywołać zamek na miejsce ofiary i aby zaabsorbował ją.}
					\item \larpcardlinepower{Transformacja}{Zniekształcenie}{\larpcardlinepassive{8}}{Pobudzenie pozwala się podszyć pod kogoś innego. Jeśli wróg zda test, to odkrywa twój kamuflaż.}
					\item \larpcardlinepower{Transformacja}{Jedność z ziemią}{Pobudzenie}{Pozwala wtopić się w ziemię i wyjść w innym miejscu zamku, niejako się teleportując.}
					\item \larpcardlinepower{Transformacja}{Kształtowanie ciała}{\larpcardlineversus{6}{Siła}}{Pozwala ci na zmianę czyjegoś wyglądu, aby wzbudzić innych nieufność albo podszyć.}
					\item \larpcardlinepower{Dominacja}{Zapomnienie}{\larpcardlineversus{5}{Siła Woli}}{Możesz nadpisać czyjeś wspomnienia, aby myślał że jest kimś innym.}
					\item \larpcardlinepower{Transformacja}{Pierwotna broń}{Pobudzenie}{Możesz zmienić swoje ciało, aby przypominało czyjąś broń, zadaje zawsze \larpcardlineweapon{+0} ciężkich obrażeń.}
				\end{itemize}
			}
			{
				\begin{itemize}[noitemsep]
					\item Chronić kapitana tak, aby nie wzbudzać podejrzeń.
					\item Dać zaabsorbować lub przerobić wszystkich innych, a na końcu kapitana.
					\item Nie dać się rozpoznać.
					\item Przeżyć.
				\end{itemize}
			}{
			\larpcardlineattributes{7}{4}{4}}
			{}
			{\larpcardlinepower{Transformacja}{Swobodne serce}{\larpcardlinepassive{7}}{Twoje serce wędruje po ciele, przez co trzeba wykonać dodatkowy test, jeśli ktoś chce przebić je kołkiem.}}
			
		\thispagestyle{empty}
		\larpcardcharacter{Siostra Anastazja}{
			\larpcardlinespecies{Człowiek}{Siostra agentka}{Watykan}}{
			Inkwizycja nie ma najlepszej renomy w Watykanie.
			Dlatego sam papież posłał ciebie, abyś kontrolowała postępy w niszczeniu zamku.
			Posiadasz pistolet na wodę święconą.
			Jesteś bezwzględna w wykonywaniu swojej pracy i nie masz litości dla pomiotów szatańskości.
			Jeżeli przypałętał się do was dziwny człowiek, nie pozwól aby dołączył i wzmocnił Inkwizycję.
			Nosisz strój z rurkami wypełnionymi wodą święconą.
			Posiadasz mini-paschał i zapalarkę, co pozwala ci poświęcić kubek wody. Aby poświęcić wodę, musisz zapalić paschał, umieścić w wodzie, odmówić krótką modlitwę, wyciągnąć i zdmuchnąć. Dla większych zbiorników trzeba to powtarzać.
			Możesz wylać na dowolnego człowieka kubek wody święconej, aby uleczyć jedną powierzchowną ranę.
			}
			{
				\begin{itemize}[noitemsep]
					\item \larpcardlinepower{Broń}{Gaszenie pożaru}{\larpcardlineweapon{+3}}{Twój malutki pistolecik jest wstanie ugasić pożar.}
					\item \larpcardlinepower{Wyposażenie}{Wzniecanie pożaru}{\larpcardlinedifficulty{6}{3}}{Potrafisz wzniecać pożary w przejściach. Mają taką siłę, ile dodatkowych punktów sukcesu ci się uda wyrzucić. Zapisz tą wartość w kratce.}
					\item \larpcardlinepower{Broń}{Pistolet na wodę święconą}{\larpcardlineweapon{+2}}{Pistolet zadaje lekkie obrażenia wampirom i ciężkie duchom oraz zmniejsza cechy wampirom. Możesz opryskać kogoś na odległość.}
					\item \larpcardlinepower{Wyposażenie}{Rurki z wodą święconą}{\larpcardlinepassive{7}}{Jeśli ktoś zaczyna cię pić, rzuca czy udało mu się poprawnie wgryźć. Inaczej napije się wody święconej, co zabiera mu punkt krwi.}
					\item \larpcardlinepower{Wyposażenie}{Krochmalony habit}{\larpcardlinepassive{3}}{Masz małą szansę na negację zadanych ci obrażeń.}
					\item \larpcardlinepower{Broń}{Włócznia z relikwią JPII}{\larpcardlineweapon{+2}}{Włócznia zadaje ciężkie obrażenia wampirom.}
				\end{itemize}
			}
			{
				\begin{enumerate}[noitemsep]
					\item Zabić wszystkich nieludzi.
					\item Upewnić się, że Ojciec Arnold wykonuje poprawnie swoją pracę.
					\item Przekonać Ludwika, aby sobie poszedł i nie dołączał do Inkwizycji.
					\item Zniszczyć zamek.
				\end{enumerate}
			}{
			\larpcardlineattributes{5}{6}{4}}
			{}
			{\larpcardlinepower{Zakonnictwo}{Odzyskanie dziewictwa}{Pasywna}{Odzyskałeś swoje dziewictwo/prawniczkowatość. Było warto.}}
			
		\thispagestyle{empty}
		\larpcardcharacter{Obrońca Włości}{
		\larpcardlinespecies{Potwór}{Bestia przyzwana przez Księcia}{Dworzanie Księcia}}{
			Jesteś wielkim potworem o 4 łapach, 4 oczach i paszczy z wielkimi zębiskami.
			Zostałeś przyzwany nekromancją przez Księcia abyś strzegł jego domeny.
			Lubisz łaskotki po brzuszku.
			Nie umiesz się prosto porozumiewać z innymi istotami, chyba że mają odpowiednie moce.
			Bardzo przeszkadzają ci wysokie piski i uciekasz od nich.
			Masz 5 kolców, które możesz wystrzelić. Masz 5 jaj, które możesz w kimś złożyć.
			Możesz kogoś pożreć, aby zabrać mu jeden punkt życia lub krwi.
			Za punkt krwi możesz zregenerować kolec.
			Trawienie kogoś trwa długo, a twoja ofiara pozostaje świadoma.
			Po przekazaniu jednego życia lub krwi może próbować się wydostać.
			Jeśli znajdziesz kogoś martwego lub w letargu, możesz w nim złożyć jaja, aby po krótkiej chwili zmienił się w twojego całkowitego poddanego.
			Poddany traci wszystkie moce, jego siła rośnie do 10, potrafi jedynie rozszarpywać pazurami, zadając poważne obrażenia \larpcardlineweapon{+2}.
			}
			{
				\begin{itemize}[noitemsep]
					\item \larpcardlinepower{Potworzastość}{Vore}{\larpcardlineversus{9}{Siła}}{Po złapaniu kogoś możesz próbować go pożreć w całości.}
					\item \larpcardlinepower{Potworzastość}{Absorpcja}{\larpcardlineversus{4}{Siła}}{Możesz absorbować krew lub punkty życia twojej ofiary w żołądku.}
					\item \larpcardlinepower{Potworzastość}{Twarde gardło}{\larpcardlinepassive{5}}{Jeśli ktoś próbuje wydostać się z twojego żołądka, musi zdać test.}
					\item \larpcardlinepower{Potworzastość}{Kolec}{\larpcardlineversus{6}{Siła}}{Strzelasz kolcem i zadajesz poważne obrażenia \larpcardlineweapon{+1}.}
					\item \larpcardlinepower{Potworzastość}{Ryk}{\larpcardlineweapon{+2}}{Pozwala ryknąć i zdmuchnąć ogień.}
					\item \larpcardlinepower{Potworzastość}{Złożenie jaj}{\larpcardlinedifficulty{6}{3}}{Składasz jaja w kimś martwym lub w letargu. Liczba sukcesów oznacza ilość żyć twojego podopiecznego.}
				\end{itemize}
			}
			{
				\begin{itemize}[noitemsep]
					\item Uratować Księcia.
					\item Chronić dworzan Księcia, pozbyć się wszystkich innych.
					\item Usunąć dziwne zjawiska z zamku.
					\item Zjeść Czerwonego Kapturka.
					\item Zrobić sobie przynajmniej jednego kolegę.
					\item Przeżyć.
				\end{itemize}

			}
			{
			\larpcardlineattributes{7}{4}{6}}
			{\larpcardlinecompulsion{Dosyć tej nienawiści! Masz ochotę każdego przytulić i polizać. Każdy będzie \emph{zmuszony} być twoim przyjacielem.}}
			{\larpcardlinepower{Potworzastość}{Kolce z rąk}{\larpcardlineversus{6}{Siła}}{Potrafisz strzelać kolcami z rąk. Kolec zadaje \larpcardlineweapon{+0} poważnych obrażeń.}}
		
		\thispagestyle{empty}
		\larpcardcharacter{Otto}{
			\larpcardlinespecies{Wampir Gangriel}{Ogrodnik zamku}{Dworzanie Księcia}}
			{
			Chcesz sadzić roślinki i tylko to cię interesuje, niech wszyscy inni wynoszą się z mojego bagna.
			Posiadasz bogaty zestaw 5 osikowych kołków, nadają się zarówno do podtrzymywania pomidorów, jak i do wbijania pijawkom w serca.
			Chcesz także zakuć Obrońcę Włości w obrożę i przywiązać na podwórku, aby strzegł rdestu. Masz już nawet dla niego budę, większą od twojej szopy.
			W twoim ciele mieszka rój biedronek, których używasz aby wyjadały mszyce na twoich kwiatkach.
			Twoje biedronki są odpędzane przez wysokie dźwięki, jako właściciel ty rzucasz rzuty obronne na to.
			}
			{
				\begin{itemize}[noitemsep]
					\item \larpcardlinepower{Animalizm}{Dzikie szepty}{Pobudzenie}{Możesz przez minutę bez problemu rozmawiać ze zwierzętami i innymi dzikusami.}
					\item \larpcardlinepower{Broń}{Łańcuch}{\larpcardlineweapon{+2}}{Gruby łańcuch/smycz do zakucia Obrońcy sprawdzi się także jako broń zadająca powierzchowne obrażenia.}
					\item \larpcardlinepower{Odporność}{Ignorowanie klątwy}{Pobudzenie}{Przez krótki moment stajesz się odporny na ogień i światło słoneczne.}
					\item \larpcardlinepower{Animalizm}{Nieumarły rój}{\larpcardlineweapon{+0}}{Wypuszczasz ze swojego ciała rój owadów, który zadaje powierzchowne obrażenia na odległość. Może także kogoś omotać i czasowo oślepić.}
					\item \larpcardlinepower{Transformacja}{Zmiana kształtu}{Pobudzenie}{Na kilka minut zmieniasz się w wielkiego, zielonego grubasa. Posiada dodatkowe własne 5 punktów życia. Możesz tylko mówić cytatami ze Shreka.}
					\item \larpcardlinepower{Broń}{Osikowe kołki}{\larpcardlineversus{6}{Siła}}{Wbijasz kołek w serce przeciwnika. Zakołkowany wampir traci wszystkie punkty krwi i wpada w letarg.}
				\end{itemize}
			}
			{
				\begin{enumerate}[noitemsep]
					\item Zakuć Obrońcę w łańcuch.
					\item Wywalić wszystkich, prócz dworzan Księcia poza zamek.
					\item Wypróbować nowe kołki w praktyce.
					\item Przyciąć żywopłot.
					\item Przeżyć.
				\end{enumerate}
			}
			{\larpcardlineattributes{5}{3}{7}}
			{\larpcardlinecompulsion{Dzikość, drapanie i gryzienie, możesz wypowiadać tylko pojedyncze słowa.}}
			{\larpcardlinepower{Animalizm}{Władca szczurów}{\larpcardlineversus{7}{Siła}}{Przyzywasz z zakamarków falę szczurów, która może kogoś dotkliwie pogryźć, zadając mu powierzchowne obrażenia.}}

		\thispagestyle{empty}
			\larpcardcharacter{Ludwik}{
			\larpcardlinespecies{Człowiek}{Fan zjawisk paranormalnych}{-}}
			{
			Czytałeś o wampirach i innych nadprzyrodzonych istotach tak dużo, że zawsze chciałeś się nim stać.
			Przy okazji jesteś ciekawy zamku i podejrzewasz, że nie jest to zwykła konstrukcja kamienna, a swego rodzaju żywa istota stworzona z magii.
			Masz sprzęt, który wydaje ci się, że tylko ty umiesz obsługiwać i który pozwala na określenie jednej z pozycji rdzenia, który steruje tym zamkiem.
			Masz broń przeciwko wampirom, która zadaje poważne obrażenia.
			Chcesz dać się spokrewnić przez silnego wampira, zmienić w ducha itp. i nie stracić przy tym rozumu.
			Jesteś trochę nieufny wobec nich, żeby nie dać się zwyczajnie wyssać od razu.
			Inkwizycja bardzo się tobą interesuje i chciałaby cię w twoich szeregach.
			}
			{
				\begin{itemize}[noitemsep]
					\item \larpcardlinepower{Wyposażenie}{Chaoskop}{2 min czekania}{Używasz urządzenia w pokoju, aby wykryć i zobaczyć wszystkie duchy, które będą w pokoju na koniec odliczania.}
					\item \larpcardlinepower{Wyposażenie}{Lampa ultrafioletowa}{\larpcardlineversus{6}{Siła}}{Gdy jesteś blisko wampira, możesz włączyć lampę UV, aby palić go nieżywcem i zadawać ciężkie obrażenia \larpcardlineweapon{+2}.}
					\item \larpcardlinepower{Wyposażenie}{Wysysacz krwi}{1 min czekania}{Możesz poprosić wampira, aby dał sobie wyssać krew, albo wprowadzić go w letarg. Otrzymaną jednostkę możesz komuś dać, albo użyć.}
					\item \larpcardlinepower{Wyposażenie}{Wykrywacz rdzenia}{20 min czekania i krew}{Ładujesz jednostkę Vitæ do urządzenia i nastawiasz na 20 minut. Po tym czasie obliczy ci jedną współrzędną rdzenia zamku. Poproś Mistrzów o karteczkę.}
					\item \larpcardlinepower{Wyposażenie}{Odstraszacz komarów}{\larpcardlineversus{7}{Siła Woli}}{Wysoki pisk potrafi odstraszać komary, biedronki i wielkie potwory. Jeśli ktoś steruje owadami, to on rzuca.}
					\item \larpcardlinepower{Wyposażenie}{Kamerka}{Pasywna}{Oglądasz świat przez kamerkę. Możesz bez trudu zobaczyć ukrywające się wampiry i przejrzeć przez fatamorganę.}
					\item \larpcardlinepower{Zabiegi}{Domieszkowana krew}{\larpcardlinepassive{8}}{Sporządziłeś wywar ze starych przepisów Inkwizycji i wlałeś go sobie w żyły. To było durne posunięcie, ale dzięki temu jeśli ktoś wgryzie ci się w żyłę i nie zda dodatkowego testu, to zostaje zobrzydliwiony i zaczyna rzygać na boki.}
				\end{itemize}
			}
			{
				\begin{itemize}[noitemsep]
					\item Przekonać kogoś aby zmienił cię w istotę nadprzyrodzoną lub dołączyć do jakiejś fajnej organizacji.
					\item Zachować swój własny umysł.
					\item Przeżyć.
				\end{itemize}
			}
			{\larpcardlineattributes{4}{7}{4}}
			{}
			{\larpcardlinepower{Zabiegi}{Domieszkowane zmutowane Vitæ}{\larpcardlineversus{5}{Siła Woli}}{Ktokolwiek wypił od teraz twoją krew, natychmiast podpada pod Więzy Krwi. Możesz mu wydawać polecenia, które musi spełnić. Możesz powtarzać test tyle razy, ile jednostek z ciebie wypił.}}
		
		
		% Bartłomiej Bratowicz, naturalny ghul, zły
% Naturalny ghul, którego ród brata się z Sabatem i wszystkim co najgorsze. Bezwzględny i obleśny. Chcesz się zemścić na Duczeskich za odwrócenie się od Tzimisce oraz na podwładnych Hadeona za odwrócenie się od ciebie. Posiadasz bronie dystansowe i białe bronie na wampiry. Masz obleśny koc, którym się ubierasz, którego nawet pożar nie chce trawić.
% Gaszenie pożaru (5)
% Przeżyć
% Pokropić rdzeń własną krwią
% Znaleźć sobie jakiegoś poddanego albo kompana do rządzenia
% Zabić wszystkich innych w najokrutniejszy sposób
% Doprowadzić kogoś do załamania nerwowego
		
		
% Sebastian, wampir, Brujah, Anarchiści
% Byłeś dobrym kolegą Księcia mimo że on należał do Camarilli. Przyszedłeś go odwiedzić, aby sprawdzić jak się miewa. Uchodzisz wśród innych poddanych za durnego dresa i trochę nim jesteś. Ale za to potrafisz szybko biegać i się wspinać. Jebać Inkwizycję na 100%. Jesteś waleczny, odważny i skłonny do układów. Potrafisz rozniecać pożary, ale za każdym razem testujesz, czy nie uciekniesz ze strachu.
% Wzniecanie pożaru (6) - siła pożaru to liczba sukcesów
% Uratować Księcia
% Przejąć kontrolę nad zamkiem, jakoś
% Kompulcja: bunt przeciwko swoim celom i narzucaniu woli przez innych
			
%TODO

% Horacy, wampir, Tzimisce, agent Hadeona Jarosławicza
% Jesteś agentem wysłanym przez samego Hadeona w celu zbadania sprawy zamku i rozszerzenia ewentualnych włości. Wiesz, że zamek jest wielkim Kniaziem stworzonym dawno temu przez potęńnego wojewodę z twojego klanu. Rozpoznajesz moce transformacji i typy spotkanych Szlachciców. Wiesz, że zamek posiada rdzeń, ale nie wiesz gdzie. Posiadasz fiolkę z krwią Hadeona, którą musisz pokropić rdzeń. Potrafisz zmienić swój i czyjś kształt, aby wyglądał na innego lub aby modyfikować jego anatomię. Jeśli nie uda ci się pokropić krwią Hadeona rdzenia, powinieneś napoić nią któregoś wampira, aby był jego sługą.
% Transformacja - (4)
% Przeżyć
% Dać Hadeonowi kontrolę nad zamkiem
% Spowodować wypicie krwi Hadeona przez jakiegoś ważnego wampira
% Kompulcja: Chęć posiadania najcenniejszych rzeczy w zamku

% Hans, wampir, Brujah, agent Księcia Szczecina
% Książę Szczecina jest córką Hadeona, a ty jesteś z nią związany więzami krwi. Zostałeś wysłany tutaj, aby przejąć władzę i poszerzyć domenę.​​​​​​​ Nie wiesz, czemu cię wysłali tutaj, ale masz zrobić co do ciebie należy. Umiesz rozpoznać moce Transformacji, jeśli je zobaczysz. Twój Książę także chciałby od ciebie ochrony którychkolwiek Tzimisce i użytkowników Transformacji.
% Rozprawić się z Camarillą
% Ochronić Horacego
% Kompulsja: Bunt przeciwko narzuconym zasadom

% Bonifacy, wampir, Toreador, agent Księcia Warszawy, Camarilla
% Hania Buszek bardzo by chciała upewnić się, że włości Camarilli zostaną w dobrych rękach.​​​​​​ Słyszałeś, że prócz syna księcia także inni sobie ostrzą kły na ową posiadłość. Zamek powinien być w niezniszczonym stanie, aby nowi lokatorzy mogli bezpiecznie w nim przebywać.
% Rozprawić się z kimkolwiek spoza Camarilli, kto chce przejąć zamek
% Ochronić mury zamku, aby był bezpieczny dla nowych lokatorów
% Kompulsja: Obsesja na punkcie czegoś




