Ten dokument musi być trzymany w tajemnicy przed graczami.
Każdy może mieć wgląd jedynie do własnej karty postaci.

\section{Fabuła}
	Zamek lokalnego Księcia Camarilli płonie.
	Poddani zwabieni łuną pożaru próbują dowiedzieć się co się dzieje.
	Jednakże zamek skrywa wiele tajemnic, i nie jesteście jedynymi zainteresowanymi.
	
	Książę Bożydar Szymonowicki został wciągnięty w czeluście zamku i fajnie było by go uratować.
	
\section{Mapa}
	Najlepiej jeśli mapa przypomina graf z niewielką ilością cykli.
	Niektóre ścieżki mogą być jednokierunkowe.
	Również należy stworzyć i oznaczyć ścieżki jako tajne przejścia, przez które tylko niektórzy mogą się przedostawać.
	
	\subsection{Kniaź}
		Zamek jest wielkim żywym Kniaziem, stworzonym dawno temu przez potężnego wojewodę Tzimisce.
		Nie wszyscy gracze o tym wiedzą.
		Potrafi atakować graczy snujących się po nim.
		Posiada rdzeń w tajnym miejscu, który steruje wszystkim.
		
		Zamek nie atakuje kapitana, chyba że przypadkiem ponieważ używa go jako przynęty.
		Kapitan został tymczasowo porwany na początku rozgrywki.
		
		Aby odkryć rdzeń, należy znaleźć jego 3 różne współrzędne \ref{sec:wspolrzedne}.
		Po znalezieniu 3 współrzędnych, gracze dowiadują się, że rdzeń się porusza i należy znaleźć jeszcze jedną wpółrzędną.
		Po znalezieniu 4 współrzędnych okazuje się, że rdzeń jest wtopiony w ciało kapitana, który o tym nie wie.
		Musiało się to stać, gdy został tymczasowo porwany.
		Jeśli kapitan nie żyje, rdzeń będzie kolejno w Szlachcicu Alfa, Potworze lub jakimś nowo-powstałym bojowym Szlachcicu.
		
		Zamek atakuje graczy.
		W kilku miejscach są poustawiane kupki z kartami \ref{sec:zamek}.
		Gracze przechodząc obok kupki biorą kartę z wierzchu i sprawdzają, co się dzieje.

	\subsection{Szlachcice}
		Wśród zakamarków i tajnych przejść czasami snują się Szlachcice, które niegdyś dbały o zamek.
		Mieszkańcy zamku są ich nieświadomi, gdyż zawsze siedzą w zakamarkach i kryją się przed wzrokiem.
		Wampiry mogą się ich napić, aby odzyskać punkty krwi.
		
		W tajnych przejściach są wyłożone karteczki z listami szlachciców.
		Gracze odkrywają karteczkę przy przejściu przez korytarz i sprawdzają na co się natknęli.
		Aby wykorzystać szlachcica do własnych celów, można użyć wampirzych mocy.
		Mogą także zostać zaatakowanym przez Zamek.
		
		Karty szlachciców, które można spotkać znajdują się w \ref{sec:szlachta}.
		Należy je połączyć z kartami Zamku \ref{sec:zamek}.
		
	\subsection{Inkwizycja}
		Zakon Św. Leopolda był świadomy prawdy o zamku, więc postanowił wykorzystać sytuację, aby pozbyć się problemu raz na zawsze.
		Wykorzystują moździerze aby zawalać zamek.
		Co jakiś czas Mistrzowie Larpu losują przejście, które zostanie zawalone.
		Gracze muszą spędzić czas na odgruzowywaniu, jeśli chcą przejść dalej.
		Zamek leczy się samoczynnie po pewnym czasie i naprawia zawały.
		Aby ustawić zawał, należy zablokować przejście na mapie i napisać karteczkę o zawale.
		Obok zostawić sznurek z supłami, które gracze muszą rozwiązać, aby odgruzować przejście.

	\subsection{Pożary}
		Zamek się także pali. Na początku na mapie jest sporo pożarów, które blokują przejście.
		Po pewnym czasie znikają.
		Pożar można ugasić wodą, przebiec przez niego itp.
		Wampiry muszą testować siłę woli na to.

\section{Narzędzia}
	Lista narzędzi, które warto mieć:
	\begin{enumerate}
		\item 5 monet krwi dla każdego gracza czerwono-czarnych -- jako licznik krwi i do przeprowadzania testów wzbudzenia.
		\item 10 monet cechy dla każdego gracza -- do przeprowadzania testów.
		\item Licznik Życia -- zestaw karteczek lub kratek na karcie do zamalowywania w stanach pełnego życia, lekkiego obrażenia i ciężkiego obrażenia.
		\item Licznik Siły Woli -- tak samo jak Życie
		\item Karty do budowania domków z kart -- dla gracza Duczeskiego.
		\item Pistolet na wodę -- dla gracza Siostry Zakonnej.
		\item Srebrne/ogniste bronie przeciwko wampirom.
		\item Czapka ze sznurkiem, prześcieradło, biała maska -- dla gracza Ducha. Zakładając na kogoś czapkę, symbolizuje jego opętanie.
			Prześcieradło aby jasno rozróżnić bohatera. Maska może być ubierana i ściągana, aby odgrywać pojawianie się ducha dla widoku dla innych.
		\item Sznurek powiązany w supły -- mechanika odgruzowywania, można zastąpić czymś, co wstrzyma graczy na pewien czas.
		\item Krótkofalówka ze słuchawkami lub inny komunikator -- dla gracza Ojca Zakonnego oraz Szlachcica Alfa do komunikacji z Mistrzami Gry.
		\item Hełm z zakrytą twarzą -- dla gracza Szeryfa i dla Szlachcica Alfa, do podszywania się.
			Drugi kask powinien mieć tekst opisujący że inni gracze są oszukani wyglądem.
		\item Jajka ugotowane na twardo i obrane -- dla Obrońcy Włości, aby odgrywać składanie jaj w graczach. Inny gracz będzie musiał zjeść to jajko w całości.
		\item Kołki drewniane -- dla Ogrodnika, jako broń. Niech przebija serca i niech nie wykole nikomu oczu.
	\end{enumerate}
	
\section{Mechanika}
	Mechaniki wspólne:
	\larplistall{}
	Mechaniki użytkowników krwi:
	\larplistblood{}
	Mechaniki wampirów:
	\larplistvampire{}
	%TODO
% Wampir pozbawiony Siły Woli szaleje i wpada w depresję. Wywołuje się jego kompulsja.
% Wampir oblany wodą święconą traci po jednym współczynniku cech.
% Duch pokropiony wodą święconą dostaje poważne obrażenie życia.
% Opętany przez ducha pokropiony wodą święconą odzyskuje kontrolę nad sobą.
% Ghule mają licznik krwi, którą mogą sobie leczyć rany i nie wpadają w szał bez niej.
% Naturalne ghule produkują jednostkę krwi kiedy zdadzą wszystkie monety.
% Wampiry bez krwi wpadają w szał i chcą kogoś wyssać.
% Pobudzenie krwi polega na rzucie monetą krwi, którą się traci jeśli nie wypadnie pozytywna strona.
% Wampir pobudza krew, aby wyleczyć jedno powierzchowne obrażenie.
% Wampir pobudza krew, aby dodać sobie na jeden test 1 punkt cech podstawowych.
% Wampir pobudza krew przy niektórych mocach.
% Srebrna broń zadaje poważne obrażenia wampirom i powierzchowne duchom.
% Normalna broń zadaje powierzchowne obrażenia wampirom i ciężkie śmiertelnikom (nie bawimy się w dzielenie na pół).
% Jeśli ktoś się w coś zmieni i nie może tego ukryć, musi nosić jakąś karteczkę albo oznajmiać innym kim jest gdy się spotkają.
% Jeśli martwa postać umrze ponownie, to jest definitywny koniec dla niej.
% Pożar ma jakąś moc, dla gaszenia należy mieć liczbę sukcesów równą lub większą niż on, dla zachowania zimnej krwi przez wampiry należy tak zdać SW.
% Po wbiciu kołka w serce wampir traci punkty krwi aż do jednego i wpada w letarg aż kołek nie zostanie usunięty.
% Po wyjściu z letargu wampir pobudza krew i jeśli mu się nie uda, wpada w szał.

	
		
\section{Karty}
	Wydrukować i pociąć na kawałki.
	\subsection{Szlachta}
	\label{sec:szlachta}
		\begin{multicols}{2}
			\larpcardenemy{Szlachcic sprzątający}{
				Staje przed wami kreatura w zjedzonym przez mole stroju pokojówki.
				Zamiast jednej dłoni ma szufelkę z własnych kości, a zamiast drugiej szczoteczkę z kręconych ludzkich włosów.
				Przepycha kurz z miejsca na miejsce i rozmazuje pajęczyny.}
				{3}{4}{1}
		\end{multicols}
		%TODO
% 		Szlachcic deratyzacyjny - mało groźny
% 		Szlachcic towarowy - można użyć do pomocy w odgruzowywaniu
% 		Szlachcic obronny - duże wyzwanie
% 		Szlachcic obliczeniowy - potrafi obliczyć jedną ze współrzędnych rdzenia
	
	\subsection{Zamek}
	\label{sec:zamek}
		\begin{multicols}{2}
			\larpcardenemy{Ręce wychodzące z murów}{
			Ze ścian wychodzą ręce i próbują złapać wszystko, co się rusza.}
			{4}{3}{1}
		\end{multicols}

% 	Ręce - ze ścian pojawiają się ręce i próbują złapać co się da (walka ≥ 3, unik ≥ 4)
% Paszcza w podłodze
% Kolce ze ścian
% Miażdżące ściany
% Rycerz - ze ściany wychodzi rycerz i walczy, jest połączony nitką ze ścianą
% Przynęta - na środku pojawia się coś bardzo cennego dla osoby, która odkrywa karteczkę, a jeśli dotknie tego, zostanie pożarta
% Fala kwasu - zadaje dodatkowe obrażenia jeśli oblany straci życie następnym razem
% Grząski teren - zamek wciągnie i zada obrażenia tym, który nie przebiegną wystarczająco szybko
% Zakryte przejście - przed wami ściana i tylko wyczuleni mogą wykryć że jest sztuczna, nic nie robi
% Jęzor - wyskakuje jęzor i atakuje tego, kto wyrzuci najmniejszą liczbę sukcesów na fizyczne

	\subsection{Współrzędne}
	\label{sec:wspolrzedne}
		\begin{multicols}{3}
			\larpcarditem{Współrzędna Rdzenia}{$\varLambda \varUpsilon \varPsi$}
			\larpcarditem{Współrzędna Rdzenia}{$\varOmega \varTheta \varGamma$}
			\larpcarditem{Współrzędna Rdzenia}{$\varDelta Z \varXi$}
			\larpcarditem{Współrzędna Rdzenia}{$T \varPi \varDelta$}
			\larpcarditem{Współrzędna Rdzenia}{$A K \varPhi$}
		\end{multicols}

	\subsection{Postaci}
	\label{sec:characters}
		\larpcardcharacter{Teodor Szymonowicki}
			{\larpcardlinespecies{Wampir Ventrue}{Dziedzic i syn Księcia}{Dworzanie Księcia, Camarilla}}{
			Twój ojciec, Książę, pałał się Nekromancją.
			Obrzydzało cię to niemiłosiernie.
			Robiłeś wszystko, aby pewnego razu jego praktyki zemściły się na nim, a ty będziesz mógł zostać jego następcą.
			Twój ojciec wydziedziczył twojego brata i nie wiesz, za co.
			Kiedyś zdiabolizowałeś Malkawianina i bardzo pragniesz to ukryć.
			Trenowałeś malkawiańskie praktyki obłędu, które w trakcie kłótni przypadkowo użyłeś na ojcu, przez co wpadł w szał i podpalił zamek.}
			{
				\begin{itemize}[noitemsep]
					\item \larpcardlinepower{Niewidoczność}{Niewidoczne przejście}{2}{Pasywna}{Pozwala na niezauważalne przejście bokiem jeśli nikt cię nie widzi.}
					\item \larpcardlinepower{Niewidoczność}{Zasłona cienia}{1}{Pobudzenie}{Wtapiasz się w tło i cię nie widać.}
					\item \larpcardlinepower{Dominacja}{Obłęd}{2}{\larpcardlineversus{7}{Siła Woli}}{Nieświadomie zadajesz powierzchowne obrażenia Siły Woli.}
				\end{itemize}

			}{
				\begin{enumerate}[noitemsep]
					\item Zwalić winę za obłęd Księcia na kogoś innego.
					\item Przeżyć.
					\item Zostać następcą Księcia.
				\end{enumerate}
			}
			{\larpcardlineattributes{6}{8}{6}}
			
		\larpcardcharacter{Sławomir Szymonowicki}{
			\larpcardlinespecies{Wampir Ventrue}{Wydziedziczony syn Księcia}{Dworzanie Księcia, Camarilla}}{
			Twój ojciec wyklął ciebie, gdy zacząłeś lepiej sobie radzić w mocach Nekromancji, niż on sam.
			Możesz użyć mocy, aby zobaczyć duchy w danym miejscu.
			Nie lubisz swojego brata i nie możesz pozwolić, aby przejął władzę.
			Chcesz aby władza została w rękach Camarilli.}
			{}{
				\begin{enumerate}[noitemsep]
					\item Zemścić się na ojcu.
					\item Znaleźć prawdę co się stało.
					\item Powstrzymać brata przed przejęciem władzy.
					\item Ukrywać swoje moce przed bratem.
					\item Przeżyć.
				\end{enumerate}
			}{
			\larpcardlineattributes{5}{7}{5}}
			
		\larpcardcharacter{Gerwazy}{
			\larpcardlinespecies{Wampir Malkawian}{Klucznik zamku}{Dworzanie Księcia, Camarilla}}{
			Jesteś zaufanym klucznikiem, znasz wszystkie tajne przejścia w zamku.
			Zawsze widzisz nieistniejące rzeczy i nie odróżniasz prawdy od własnych urojeń.
			Jesteś świadomy swojej klątwy.
			Znasz wszystkie dziwactwa występujące w zamku, ale zawsze brałeś je za urojenia.
			Z tego powodu nigdy nie powiedziałeś o niczym Księciu.}
			{}{
				\begin{enumerate}[noitemsep]
					\item Uratować Księcia.
					\item Przeżyć.
				\end{enumerate}				
			}{
			\larpcardlineattributes{4}{6}{4}}
		
