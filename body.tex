Ten dokument musi być trzymany w tajemnicy przed graczami.
Każdy może mieć wgląd jedynie do własnej karty postaci.

\section{Fabuła}
	Zamek lokalnego Księcia Camarilli płonie.
	Poddani zwabieni łuną pożaru próbują dowiedzieć się co się dzieje.
	Jednakże zamek skrywa wiele tajemnic, i nie jesteście jedynymi zainteresowanymi.
	
	Książę Bożydar Szymonowicki został wciągnięty w czeluście zamku i fajnie było by go uratować.
	
\section{Mapa}
	Najlepiej jeśli mapa przypomina graf z niewielką ilością cykli.
	Niektóre ścieżki mogą być jednokierunkowe.
	Również należy stworzyć i oznaczyć ścieżki jako tajne przejścia, przez które tylko niektórzy mogą się przedostawać.
	
	\subsection{Kniaź}
		Zamek jest wielkim żywym Kniaziem, stworzonym dawno temu przez potężnego wojewodę Tzimisce.
		Nie wszyscy gracze o tym wiedzą.
		Potrafi atakować graczy snujących się po nim.
		Posiada rdzeń w tajnym miejscu, który steruje wszystkim.
		
		Zamek nie atakuje kapitana, chyba że przypadkiem ponieważ używa go jako przynęty.
		Kapitan został tymczasowo porwany na początku rozgrywki.
		
		Aby odkryć rdzeń, należy znaleźć jego 3 różne współrzędne \ref{sec:wspolrzedne}.
		Po znalezieniu 3 współrzędnych, gracze dowiadują się, że rdzeń się porusza i należy znaleźć jeszcze jedną wpółrzędną.
		Po znalezieniu 4 współrzędnych okazuje się, że rdzeń jest wtopiony w ciało kapitana, który o tym nie wie.
		Musiało się to stać, gdy został tymczasowo porwany.
		Jeśli kapitan nie żyje, rdzeń będzie kolejno w Szlachcicu Alfa, Potworze lub jakimś nowo-powstałym bojowym Szlachcicu.
		
		Zamek atakuje graczy.
		W kilku miejscach są poustawiane kupki z kartami \ref{sec:zamek}.
		Gracze przechodząc obok kupki biorą kartę z wierzchu i sprawdzają, co się dzieje.

	\subsection{Szlachcice}
		Wśród zakamarków i tajnych przejść czasami snują się Szlachcice, które niegdyś dbały o zamek.
		Mieszkańcy zamku są ich nieświadomi, gdyż zawsze siedzą w zakamarkach i kryją się przed wzrokiem.
		Wampiry mogą się ich napić, aby odzyskać punkty krwi.
		
		W tajnych przejściach są wyłożone karteczki z listami szlachciców.
		Gracze odkrywają karteczkę przy przejściu przez korytarz i sprawdzają na co się natknęli.
		Aby wykorzystać szlachcica do własnych celów, można użyć wampirzych mocy.
		Mogą także zostać zaatakowanym przez Zamek.
		
		Karty szlachciców, które można spotkać znajdują się w \ref{sec:szlachta}.
		Należy je połączyć z kartami Zamku \ref{sec:zamek}.
		
	\subsection{Inkwizycja}
		Zakon Św. Leopolda był świadomy prawdy o zamku, więc postanowił wykorzystać sytuację, aby pozbyć się problemu raz na zawsze.
		Wykorzystują moździerze aby zawalać zamek.
		Co jakiś czas Mistrzowie Larpu losują przejście, które zostanie zawalone.
		Gracze muszą spędzić czas na odgruzowywaniu, jeśli chcą przejść dalej.
		Zamek leczy się samoczynnie po pewnym czasie i naprawia zawały.
		Aby ustawić zawał, należy zablokować przejście na mapie i napisać karteczkę o zawale.
		Obok zostawić sznurek z supłami, które gracze muszą rozwiązać, aby odgruzować przejście.

	\subsection{Pożary}
		Zamek się także pali. Na początku na mapie jest sporo pożarów, które blokują przejście.
		Po pewnym czasie znikają.
		Pożar można ugasić wodą, przebiec przez niego itp.
		Wampiry muszą testować siłę woli na to.

\section{Walka}
	Sposób opisu rozwiązywania konfliktów siłowych.
	
	Każda postać przy konflikcie dysponuje pulą punktów zapisanych przy używanej mocy lub w parametrze \textbf{Siła}, jeśli używa niestandardowych zagrywek.
	Obrona przeciwko czemuś jest zazwyczaj z parametru \textbf{Siła}, \textbf{Siła Woli}, bądź bardzo rzadko z pasywnej mocy.
	
	Punkty Siły Woli są brane z ilości pustych kratek w liczniku.
	
	Obie strony rzucają w kubku, bądź w dłoniach liczbę monet w ilości cechy.
	Różnica w liczbie zwycięstw na korzyść atakującego oznacza liczbę zadanych obrażeń, do której dodaje się siłę broni.
	Zwycięstwa na korzyść atakowanego oznaczają, że udało się jej uniknąć konsekwencji, chyba że posiada odpowiednią pasywną moc do kontrataku.
	
	Jeśli ktoś chce kogoś zaatakować wręcz, to musi takie osoby wpierw pochwycić lub zagonić w kozi róg, chyba że jest niewidzialny, bądź złapie je fizycznie z zaskoczenia.
	Atak dystansowy nie wymaga pochwycenia, a atakowana osoba jest informowana o tym, że atakujący w nią strzela.
	Jeśli jest kilka osób, które chcę walczyć ze sobą, to rzucają do skutku z parametru \textbf{Siła} i kto będzie miał najwięcej zwycięstw, zaczyna atak.
	Osoba atakująca ma dodatkową monetę, jeśli ktoś jej pomaga w ataku.
	Jeśli celem ataku jest kilka osób, wtedy należy podzielić swoją pulę punktów i zaatakować każdego osobno jej częścią.
	
	Atakujący zadaje obrażenia, osoba atakowana prawie nigdy nie zadaje obrażeń.
	
	Po wykonaniu akcji, atakowana osoba może próbować uciekać bądź atakować agresora z powrotem.

\section{Narzędzia}
	Lista narzędzi, które warto mieć:
	\begin{enumerate}
		\item 5 monet krwi dla każdego gracza czerwono-czarnych -- jako licznik krwi i do przeprowadzania testów wzbudzenia.
		\item 10 monet cechy dla każdego gracza -- do przeprowadzania testów.
		\item Licznik Życia -- zestaw karteczek lub kratek na karcie do zamalowywania w stanach pełnego życia, lekkiego obrażenia i ciężkiego obrażenia.
		\item Licznik Siły Woli -- tak samo jak Życie
		\item Karty do budowania domków z kart -- dla gracza Duczeskiego.
		\item Pistolet na wodę -- dla gracza Siostry Zakonnej.
		\item Srebrne/ogniste bronie przeciwko wampirom.
		\item Czapka ze sznurkiem, prześcieradło, biała maska -- dla gracza Ducha. Zakładając na kogoś czapkę, symbolizuje jego opętanie.
			Prześcieradło aby jasno rozróżnić bohatera. Maska może być ubierana i ściągana, aby odgrywać pojawianie się ducha dla widoku dla innych.
		\item Sznurek powiązany w supły -- mechanika odgruzowywania, można zastąpić czymś, co wstrzyma graczy na pewien czas.
		\item Krótkofalówka ze słuchawkami lub inny komunikator -- dla gracza Ojca Zakonnego oraz Szlachcica Alfa do komunikacji z Mistrzami Gry.
		\item Hełm z zakrytą twarzą -- dla gracza Szeryfa i dla Szlachcica Alfa, do podszywania się.
			Drugi kask powinien mieć tekst opisujący że inni gracze są oszukani wyglądem.
		\item Jajka ugotowane na twardo i obrane -- dla Obrońcy Włości, aby odgrywać składanie jaj w graczach. Inny gracz będzie musiał zjeść to jajko w całości.
		\item Kołki drewniane -- dla Ogrodnika, jako broń. Niech przebija serca i niech nie wykole nikomu oczu.
	\end{enumerate}
	
	
\section{Mechanika}
	Lista mechanik, które przestrzegają dane postacie.
	
	Mechaniki wspólne dla wszystkich:
	\begin{itemize}
		\item Test cechy -- rzuć monetami w liczbie cechy, musisz wyrzucić liczbę sukcesów $\geqslant$ jak wróg.
		\item Obrażenia otrzymuje atakowana strona w liczbie różnic sukcesów.
		\item Liczniki Siły Woli i Życia są w formie trójwartościowych wskaźników. Brak obrażeń, obrażenie lekkie i ciężkie.
		\item Możesz przerzucić zwykłą monetę za powierzchowne obrażenie Siły Woli.
		\item Test Siły Woli -- rzuć monetami w liczbie pustych kratek Siły Woli.
		\item Możesz oddać swoje punkty życia wampirowi w letargu, aby go obudzić.
	\end{itemize}

	Mechaniki dla użytkowników krwi:
	\begin{itemize}
		\item Ilość monet krwi oznacza poziom krwi.
		\item Pobudzenie krwi polega na rzucie jedną monetą krwi i stracie jej przy czarnym wyniku.
		\item Pobudź krew, aby wyleczyć 1 powierzchownych obrażeń.
		\item Pobudź krew, aby wzmocnić się na jeden test o 1 punktów siły, można powtarzać.
		\item Przy użyciu każdej dyscypliny pobudź krew.
	\end{itemize}

	Mechaniki dla wampirów:
	\begin{itemize}
		\item Po utracie życia zapadasz w letarg i ktoś musi oddać ci punkty życia, abyś wstał.
		\item Po wyjściu z letargu pobudzasz krew i jeśli się nie uda, wpadasz w szał.
		\item Po utracie krwi twoja Bestia wpada w szał.
		\item Po utracie Siły Woli szalejesz i wpadasz w depresję. Wywołuje się twoja kompulsja klanowa.
		\item Gdy twoja Bestia wpada w szał, to chce kogoś natychmiast wyssać. Twoja siła jest podwojona na ten czas.
	\end{itemize}

	Mechaniki dla śmiertelników:
	\begin{itemize}
		\item Po utracie życia jesteś w stanie krytycznym na granicy śmierci.
		\item Po utracie Siły Woli wpadasz w depresję.
	\end{itemize}

	Mechaniki dla ghuli:
	\begin{itemize}
		\item Po utracie krwi nie wpadasz w szał.
		\item Odzyskujesz punkt krwi co 10 minut.
	\end{itemize}

	%TODO MECHANIKA
% Wampir oblany wodą święconą traci po jednym współczynniku cech.
% Duch pokropiony wodą święconą dostaje poważne obrażenie życia.
% Opętany przez ducha pokropiony wodą święconą odzyskuje kontrolę nad sobą.
% Naturalne ghule produkują jednostkę krwi kiedy zdadzą wszystkie monety.
% Srebrna broń zadaje poważne obrażenia wampirom i powierzchowne duchom.
% Normalna broń zadaje powierzchowne obrażenia wampirom i ciężkie śmiertelnikom (nie bawimy się w dzielenie na pół).
% Jeśli ktoś się w coś zmieni i nie może tego ukryć, musi nosić jakąś karteczkę albo oznajmiać innym kim jest gdy się spotkają.
% Jeśli martwa postać umrze ponownie, to jest definitywny koniec dla niej.
% Pożar ma jakąś moc, dla gaszenia należy mieć liczbę sukcesów równą lub większą niż on, dla zachowania zimnej krwi przez wampiry należy tak zdać SW.
% Po wbiciu kołka w serce wampir traci punkty krwi aż do jednego i wpada w letarg aż kołek nie zostanie usunięty.

	
		
\section{Karty}
	Wydrukować i pociąć na kawałki.
	\subsection{Szlachta}
	\label{sec:szlachta}
		\begin{multicols}{2}
			\larpcardenemy{Szlachcic sprzątający}{
				Staje przed wami kreatura w zjedzonym przez mole stroju pokojówki.
				Zamiast jednej dłoni ma szufelkę z własnych kości, a zamiast drugiej szczoteczkę z kręconych ludzkich włosów.
				Przepycha kurz z miejsca na miejsce i rozmazuje pajęczyny.
				}{
				}{3}{4}{1}
			\larpcardenemy{Szlachcic deratyzacyjny}{
				Kreatura brudna i ufajdana błotem, z której zwisają szkielety licznych zwierzątek.
				Patrzy się na was złowieszczo, abyście dołączyli do jego kolekcji.
				}{
				\larpcardlineattack{Z torsu natychmiastowo wyskakuje mu pazurzasta łapa, używana normalnie do łapania szczurów, atakuje tylko raz na początku.}{3}
				}{2}{3}{2}
			\larpcardenemy{Szlachcic towarowy}{
				Wsuwa się przed was kreatura niska i szeroka. Posiada 2 dodatkowe nogi i szerokie stopy.
				Potężne mięśnie na ramionach i olbrzymie dłonie.
				Łazi niezgrabnie z miejsca na miejsce.
				}{
				}{4}{5}{3}
		\end{multicols}
		%TODO
% 		Szlachcic obronny - duże wyzwanie
% 		Szlachcic obliczeniowy - potrafi obliczyć jedną ze współrzędnych rdzenia
	
	\subsection{Zamek}
	\label{sec:zamek}
		\begin{multicols}{2}
			\larpcardenemy{Ręce wychodzące z murów}{
				Ze ścian wychodzą ręce i próbują złapać wszystko, co się rusza.
				}{
				\larpcardlineattack{Każdy, kto próbuje przekroczyć obszar, jest łapany i rozrywany.}{4}
				}{-}{3}{1}
			\larpcardenemy{Paszcza w podłodze}{
				W podłodze otwiera się potężna paszcza z zębami.
				Nie można jej atakować, a jedynie przeskoczyć.
				}{
				}{5}{-}{-}
			\larpcardenemy{Kolce ze ścian}{
				Długie kolce wysuwają się i strzelają w kogokolwiek, kto spróbuje przejść naprzeciwko.
				Nie można ich atakować.
				Można próbować prześlizgnąć się obok.
				}{
				\larpcardlineattack{Jeśli nie udało się prześlizgnąć, kolec wyskakuje z impetem i robi szaszłyk.}{6}
				}{4}{-}{-}
		\end{multicols}
		
	

		%TODO
		% Miażdżące ściany
		% Rycerz - ze ściany wychodzi rycerz i walczy, jest połączony nitką ze ścianą
		% Przynęta - na środku pojawia się coś bardzo cennego dla osoby, która odkrywa karteczkę, a jeśli dotknie tego, zostanie pożarta
		% Fala kwasu - zadaje dodatkowe obrażenia jeśli oblany straci życie następnym razem
		% Grząski teren - zamek wciągnie i zada obrażenia tym, który nie przebiegną wystarczająco szybko
		% Płytki naciskowe - płytki da się rozróżnić i niektóre zapadają się wprost do zbiornika macek
		% Zakryte przejście - przed wami ściana i tylko wyczuleni mogą wykryć że jest sztuczna, nic nie robi
		% Jęzor - wyskakuje jęzor i atakuje tego, kto wyrzuci najmniejszą liczbę sukcesów na fizyczne

	\subsection{Karty pomocnicze bohaterów}
		Tutaj są zapisane wskazówki dla graczy, wykorzystujące ich moce.
		%TODO
		% Opis rytuału egzorcyzmu
		% Opis użycia kalkulatora współrzędnych

	\subsection{Współrzędne}
	\label{sec:wspolrzedne}
		\begin{multicols}{3}
			\larpcarditem{Współrzędna Rdzenia}{$\varLambda \varUpsilon \varPsi$}
			\larpcarditem{Współrzędna Rdzenia}{$\varOmega \varTheta \varGamma$}
			\larpcarditem{Współrzędna Rdzenia}{$\varDelta Z \varXi$}
			\larpcarditem{Współrzędna Rdzenia}{$T \varPi \varDelta$}
			\larpcarditem{Współrzędna Rdzenia}{$A K \varPhi$}
		\end{multicols}

	\subsection{Postaci}
	\label{sec:characters}
		Postacie są uporządkowane w kolejności, w jakiej należy je przyporządkowywać kolejnym osobom biorącym udział w zabawie.
		Minimalna ilość graczy to 4.
		\thispagestyle{empty}
		\larpcardcharacter{Teodor Szymonowicki}
			{\larpcardlinespecies{Wampir Ventrue}{Dziedzic i syn Księcia}{Dworzanie Księcia, Camarilla}}{
			Twój ojciec, Książę, pałał się Nekromancją.
			Obrzydzało cię to niemiłosiernie.
			Robiłeś wszystko, aby pewnego razu jego praktyki zemściły się na nim, a ty będziesz mógł zostać jego następcą.
			Twój ojciec wydziedziczył twojego brata i nie wiesz, za co.
			Kiedyś zdiabolizowałeś Malkawianina i bardzo pragniesz to ukryć.
			Trenowałeś malkawiańskie praktyki obłędu, które w trakcie kłótni przypadkowo użyłeś na ojcu, przez co wpadł w szał i podpalił zamek.}
			{
				\begin{itemize}[noitemsep]
					\item \larpcardlinepower{Niewidoczność}{Zasłona cienia}{\larpcardlineversus{4}{Siła}}{Wtapiasz się w tło i cię nie widać.}
					\item \larpcardlinepower{Dominacja}{Obłęd}{\larpcardlineversus{7}{Siła Woli}}{Nieświadomie zadajesz powierzchowne obrażenia Siły Woli.}
					\item \larpcardlinepower{Dominacja}{Mesmeryzm}{\larpcardlineversus{5}{Siła Woli}}{Rozkazujesz komuś natychmiast wykonać konkretne instrukcje.}
					\item \larpcardlinepower{Dominacja}{Podświadome instrukcje}{\larpcardlineversus{6}{Siła}}{Zapisujesz w kimś odpowiedni warunek i akcję, którą ma wykonać, w formie tekstu na karteczce: \emph{Jeśli stanie się X to zrób Y.} Druga osoba wykonuje w tajemnicy test \larpcardlineversus{5}{Siła}, czy nie udało jej się przezwyciężyć komendy.}
					\item \larpcardlinepower{Broń}{Sztylet}{\larpcardlineweapon{-1}}{Atak z bliska.}
				\end{itemize}

			}{
				\begin{enumerate}[noitemsep]
					\item Zwalić winę za obłęd Księcia na kogoś innego.
					\item Nie pozwolić, aby klucznik dowiedział się o twoim diabolizmie.
					\item Przeżyć.
					\item Zostać następcą Księcia.
				\end{enumerate}
			}
			{\larpcardlineattributes{6}{8}{6}}
			{\larpcardlinecompulsion{Arogancja, żądza władzy nad innymi.}}
			
		\thispagestyle{empty}
		\larpcardcharacter{Sławomir Szymonowicki}{
			\larpcardlinespecies{Wampir Ventrue}{Wydziedziczony syn Księcia}{Dworzanie Księcia, Camarilla}}{
			Twój ojciec wyklął ciebie, gdy zacząłeś lepiej sobie radzić w mocach Nekromancji, niż on sam.
			Możesz użyć mocy, aby zobaczyć duchy w danym miejscu.
			Nie lubisz swojego brata i nie możesz pozwolić, aby przejął władzę.
			Chcesz aby władza została w rękach Camarilli.
			Oblivion to twoja broń i nie potrzebujesz żadnej innej.}
			{
				\begin{itemize}[noitemsep]
					\item \larpcardlinepower{Nekromancja}{Wizja Obliviona}{Pobudzenie}{Potrafisz zobaczyć wszystkie duchy w pomieszczeniu.}
					\item \larpcardlinepower{Nekromancja}{Ręce Ahrimana}{\larpcardlineversus{5}{Siła}}{Macki wychodzą z cieni, poruszają się po powierzchniach, mogą atakować i manipulować rzeczami.}
					\item \larpcardlinepower{Nekromancja}{Dotyk Obliviona}{\larpcardlineversus{5}{Siła}}{Twój dotyk powoduje poważne obrażenia \larpcardlineweapon{+2}, uszkadzając ciało bądź dotykane przedmioty. Bronie tracą jeden punkt siły.}
					\item \larpcardlinepower{Dominacja}{Zatarcie wspomnień}{\larpcardlineversus{6}{Siła}}{Mówisz ofierze, jakie jej znane ci wspomnienia mają być zatarte podanymi.}
					\item \larpcardlinepower{Prezencja}{Zachwyt}{\larpcardlineversus{4}{Siła}}{Powodujesz, że ofiara na krótko zmienia nastawienie do ciebie i nie chce atakować.}
				\end{itemize}
			}{
				\begin{enumerate}[noitemsep]
					\item Zemścić się na ojcu.
					\item Znaleźć prawdę co się stało.
					\item Powstrzymać brata przed przejęciem władzy.
					\item Ukrywać swoje moce przed bratem.
					\item Przeżyć.
				\end{enumerate}
			}{
			\larpcardlineattributes{5}{7}{5}}
			{\larpcardlinecompulsion{Arogancja, chęć podporządkowania sobie całego najbliższego świata niematerialnego.}}
			
		\thispagestyle{empty}
		\larpcardcharacter{Gerwazy}{
			\larpcardlinespecies{Wampir Malkawian}{Klucznik zamku}{Dworzanie Księcia, Camarilla}}{
			Jesteś zaufanym klucznikiem, znasz wszystkie tajne przejścia w zamku.
			Zawsze widzisz nieistniejące rzeczy i nie odróżniasz prawdy od własnych urojeń.
			Jesteś świadomy swojej klątwy.
			Znasz wszystkie dziwactwa występujące w zamku, ale zawsze brałeś je za urojenia.
			Z tego powodu nigdy nie powiedziałeś o niczym Księciu.}
			{
				\begin{itemize}[noitemsep]
					\item \larpcardlinepower{Nadwrażliwość}{Fatamorgana}{\larpcardlineversus{4}{Siła}}{Tworzysz iluzje wpływające na zmysły wszystkich wokół.}
					\item \larpcardlinepower{Nadwrażliwość}{Odczytywanie duszy}{\larpcardlineversus{6}{Siła Woli}}{Odczytujesz gatunek osoby, jej cechy i liczniki, dowiadujesz się czy popełniła diabolizm.}
					\item \larpcardlinepower{Nadwrażliwość}{Przeczucie}{Pobudzenie}{Pozwala prywatnie podejrzeć pierwszą karteczkę na stosie.}
					\item \larpcardlinepower{Nadwrażliwość}{Wyczucie niewidocznego}{Pasywna}{Pozwala w przybliżeniu wyczuć niewidzialne rzeczy w okolicy.}
					\item \larpcardlinepower{Broń}{Pęk kluczy}{\larpcardlineweapon{-1}}{Pęk bardzo ostrych kluczy na sznurku jest dobrą bronią.}
				\end{itemize}
			}{
				\begin{enumerate}[noitemsep]
					\item Uratować Księcia.
					\item Przeżyć.
				\end{enumerate}				
			}{
			\larpcardlineattributes{4}{6}{4}}
			{\larpcardlinecompulsion{Urojenia i wizje.}}
			
		\thispagestyle{empty}
		\larpcardcharacter{Kamil}{
			\larpcardlinespecies{Wampir Toreador}{Kapitan Księcia}{Dworzanie Księcia, Camarilla}}{
			Jesteś zaufanym kapitanem Księcia.
			Zostałeś wciągnięty przez zamek, ale z jakichś przyczyn nadal nieżyjesz.
			Musisz pomóc absolutnie wszystkim rozwiązać ich problemy.
			Coś dziwnego się dzieje z twoim ciałem, jeśli się skupisz, i nie masz pojęcia, co.
			Możesz modyfikować swoje ręce na ostrza lub stać się zupełnie lekki.
			Lepiej nie pokazywać tych zdolności innym.
			}{
				\begin{itemize}[noitemsep]
					\item \larpcardlinepower{Transformacja}{Waga piórka}{Pobudzenie}{Zmniejszasz swój ciężar niemal do zera.}
					\item \larpcardlinepower{Transformacja}{Pierwotna broń}{Pobudzenie}{Zmieniasz ręce w ostrza i możesz atakować nimi jak srebrem \larpcardlineweapon{+4}}.
					\item \larpcardlinepower{Nadwrażliwość}{Obeah}{\larpcardlineversus{6}{2}}{Leczysz tyle powierzchownych obrażeń Siły Woli, ile masz nadwyżek punktów sukcesu.}
					\item \larpcardlinepower{Przyspieszenie}{Przebiegnięcie}{\larpcardlineversus{6}{Siła}}{Potrafisz natychmiastowo przebiec przez obszar i biegać po wodzie.}
					\item \larpcardlinepower{Broń}{Miecz}{\larpcardlineweapon{+2}}{Zwykły metalowy miecz, jakich wiele.}
				\end{itemize}
			}{
				\begin{enumerate}[noitemsep]
					\item Uratować Księcia.
					\item Dowiedzieć się prawdy o zamku.
					\item Zemścić się na tym, kto porwał Księcia.
					\item Dowiedzieć się, co to za nowe moce posiadasz.
				\end{enumerate}
			}{
			\larpcardlineattributes{5}{4}{6}}
			{\larpcardlinecompulsion{Obłęd na temat jakiejś rzeczy lub kogoś, możesz mówić tylko o tym.}}
			
		\thispagestyle{empty}
		\larpcardcharacter{Ryszard}{
			\larpcardlinespecies{Wampir Nosferatu}{Szeryf Księcia}{Dworzanie Księcia, Camarilla}}{
			Całe nieżycie byłeś razem z Księciem.
			Zwykle nosisz zakryty hełm aby wzbudzać mniejszy strach wśród wrogów waszej domeny.
			Nie pamiętasz, gdy ostatni raz widziałeś swoją twarz w lustrze, a nadal masz koszmary.
			Posiadasz zastraszanie oraz znasz się na anatomii i torturach.
			Umiesz obsługiwać zaawansowane urządzenia technologiczne.}
			{
				\begin{itemize}[noitemsep]
					\item \larpcardlinepower{Nosferatu}{Wygląd Nosferatu}{\larpcardlineversus{6}{Siła Woli}}{Pokazujesz swój upiorny wygląd i zadajesz powierzchowne obrażenia Siły Woli \larpcardlineweapon{+0}.}
					\item \larpcardlinepower{Animalizm}{Poskromienie Bestii}{\larpcardlineversus{7}{Siła}}{Powoduje apatię lub blokuje używanie mocy krwi.}
					\item \larpcardlinepower{Potencja}{Daleki skok}{Pobudzenie}{Pozwala przeskoczyć kawałek podłoża.}
					\item \larpcardlinepower{Niewidoczność}{Niewidoczne przejście}{Pobudzenie}{Pozwala na przemykanie, będąc niezauważalnym, jeśli się jest cicho i zniknęło się niezauważonym.}
					\item \larpcardlinepower{Broń}{Wyrąbisty miecz}{\larpcardlineweapon{+3}}{Ten miecz spopielił już wielu wrogów Camarilli, a nadal się nie stępił.}
					%TODO
				\end{itemize}

			}{
				\begin{enumerate}[noitemsep]
					\item Uratować Księcia.
					\item Ochronić poddanych.
					\item Ochronić mury zamku, abyście dalej mogli w nim mieszkać.
				\end{enumerate}
			}{
			\larpcardlineattributes{6}{6}{7}}
			{\larpcardlinecompulsion{Kryptofilia. Chęć poznania wszystkich możliwych sekretów wszystkiego wokół.}}
			
		\thispagestyle{empty}
		\larpcardcharacter{Amelia}{
			\larpcardlinespecies{Wampir Banu Haquim}{Tajna zabójczyni}{-}}{
			Informacja o tym, kto jest twoim zleceniodawcą, została ci wymazana z głowy.
			Wiesz że twój klan nigdy się nie poddaje, a jeśli ktoś ma być zabity, to będzie to nieuniknione.
			Twoim celem jest szeryf.
			I warto by zwalić winę na kogoś innego.
			Posiadasz zestaw broni na wampiry.
			Uważasz, że inne wampiry nie mają prawa należeć do żadnych sekt jak Camarilla, a tylko przynależność klanowa się liczy.
			}{
				\begin{itemize}[noitemsep]
					\item \larpcardlinepower{Magia Krwi}{Korozyjne Vitæ}{Pobudzenie}{Twoja krew produkuje silny kwas, jeden test pobudzenia na 1 dm³ materiału.}
					\item \larpcardlinepower{Magia Krwi}{Stłumienie Vitæ}{\larpcardlineversus{4}{Siła Woli}}{Usuwasz innemu użytkownikowi krwi jeden punkt krwi.}
					\item \larpcardlinepower{Magia Krwi}{Dotyk Skorpiona}{Pobudzenie}{Twoja krew produkuje silną truciznę, którą można pokryć broń dla podwójnych obrażeń.}
					\item \larpcardlinepower{Magia Krwi}{Łyk Skorpiona}{\larpcardlinepassive{6}}{Używasz Dotyku Skorpiona, aby wysysaną sobie krew zmienić w truciznę. Wysysający musi zdać test.}
					\item \larpcardlinepower{Magia Krwi}{Kradzież Vitæ}{\larpcardlineversus{6}{Siła}}{Rozrywasz komuś tętnicę na odległość, aby przesikać 1 punkt krwi wprost do ust. Rana się zasklepia.}
				\end{itemize}
			}
			{
				\begin{itemize}[noitemsep]
					\item Sproszkować szeryfa Księcia.
					\item Zwalić winę na kogoś innego.
					\item Uciec.
				\end{itemize}
			}{
			\larpcardlineattributes{7}{5}{4}}
			{\larpcardlinecompulsion{Osądzenie i zabójstwo każdego, kto się nie zgadza z twoimi przekonaniami.}}
			
		\thispagestyle{empty}
		\larpcardcharacter{Dionizy Duczeski}{
			\larpcardlinespecies{Naturalny ghul}{Znajomy Księcia}{Dworzanie Księcia, Camarilla}}{
			Ghule stworzone dawno temu przez Tzimisce po wielu pokoleniach nauczyły się tworzyć własne Witæ i rozmnażać jak ludzie.
			Rodzina Duczeskich jest szanowanym rodem inteligenckim.
			Wasze badania opierają się o tworzenie skomplikowanych mechanizmów zegarowych.
			Posiadasz 5 rozkładalnych klatek, które chronią przed zawałem, rozłożenie klatki wymaga postawienia domku z kart.
			Czytałeś o mocach transformacji Tzimisce i podejrzewasz, że zamek jest wielkim Kniaziem i ma rdzeń sterujący, ale nie wiesz, gdzie.
			Możesz skonstruować blokadę na rdzeń, która uśpi zamek.
			Aby skonstruować blokadę, musisz zebrać pośród zamku 5 ciekawych przedmiotów o tematyce blokowania, a potem spróbować skonstruować blokadę, pytając się Mistrza Larpu.
			Mistrz Larpu może zaakceptować albo powiedzieć, które z przedmiotów się nie nadają.}
			{
				\begin{itemize}[noitemsep]
					\item \larpcardlinepower{Wykształcenie}{Przekonywanie}{\larpcardlinepassive{7}}{Masz dar przekonywania wampirów, aby cię nie wyssali.}
					\item \larpcardlinepower{Wyposażenie}{Nakręcana zbroja}{\larpcardlinepassive{4}}{Atakujący musi zdać dodatkowy test, aby sprawdzić, czy automatyczna zbroja nie zablokowała jego ataku całkowicie.}
					\item \larpcardlinepower{Odporność}{Twardy umysł}{\larpcardlinepassive{5}}{Twoja Siła Woli używana w obronie jest wzmocniona.}
					\item \larpcardlinepower{Wyposażenie}{Automaton ratowniczy}{\larpcardlineversus{5}{3}}{Wykonaj Pobudzenie krwi, aby zasilić automaton i zdaj test, aby zgasić pożar lub użyć większej łyżki do odgruzowywania.}
					\item \larpcardlinepower{Wyposażenie}{Automaton obronny}{\larpcardlinepassive{3}}{Wykonaj pobudzenie krwi, aby zasilić automaton, który będzie chronił dowolną grupę osób. Ktokolwiek was atakuje, będzie musiał dodatkowo zdać test.}
					\item \larpcardlinepower{Broń}{Zawodny pistolet na krew}{\larpcardlineversus{3}{Siła}}{Musisz wykonać Pobudzenie krwi, aby naładować pistolet. Zadaje \larpcardlineweapon{+0}.}
				\end{itemize}
			}
			{
				\begin{enumerate}[noitemsep]
					\item Przeżyć.
					\item Uśpić zamek za pomocą blokady.
					\item Uratować wszystkie dobre istoty.
				\end{enumerate}
			}{
			\larpcardlineattributes{4}{8}{4}}
			{}
			
		\thispagestyle{empty}
		\larpcardcharacter{Ojciec Arnold}{
			\larpcardlinespecies{Człowiek}{Zakonnik Inkwizycji}{Inkwizycja}}{
			Zostałeś posłany przez Zakon Św. Leopolda, aby upewnić się, że wszystko w zamku ma być ostatecznie martwe.
			Nie podzielasz jednak wszystkich idei swoich przełożonych.
			Chcesz uratować wszystkie dobre istoty w tajemnicy przed resztą zakonu.
			Posiadasz Prawdziwą Wiarę, która odstrasza wampiry.
			Zamek jest ostrzeliwany przez Inkwizycję, masz informację o wszystkich kolejnych zawałach korytarzy.
			Wiesz, że zamek jest wielkim Kniaziem stworzony dawno temu przez wojewodę Tzimisce.
			Posiada rdzeń, który wszystko kontroluje.
			Znasz jedną ze współrzędnych rdzenia.
			Możesz wykonać egzorcyzm na kimś, aby wypędzić z niego ducha.
			Potrafisz rozpoznać moce transformacji, jeśli ktoś ich używa.}
			{
				\begin{itemize}[noitemsep]
					\item \larpcardlinepower{Wyposażenie}{Kolczuga}{Pasywna}{Wszystkie ciężkie obrażenia zamieniają się na powierzchowne.}
					\item \larpcardlinepower{Prawdziwa Wiara}{Niesmaczna krew}{Pasywna}{Kto napije się krwi, musi za każdy punkt wykonać \larpcardlineversus{10}{Siła Woli}.}
					\item \larpcardlinepower{Prawdziwa Wiara}{Odstraszenie wampirów}{\larpcardlineversus{9}{Siła Woli}}{Zmuszasz wszystkie wampiry, aby uciekły z pokoju.}
					\item \larpcardlinepower{Prawdziwa Wiara}{Wystraszenie duchów}{\larpcardlineversus{8}{Siła Woli}}{Zadajesz obrażenia Siły Woli wszystkim duchom w pokoju, jeśli w nim są}.
					\item \larpcardlinepower{Inkwizycja}{Egzorcyzm}{\larpcardlineversus{7}{Siła Woli}}{Wypędzasz ducha z opętanej osoby i zadajesz mu powierzchowne obrażenia.}
					\item \larpcardlinepower{Broń}{Pastorał}{\larpcardlineweapon{+2}}{Święcony pastorał zadaje poważne obrażenia wampirom i powierzchowne śmiertelnikom.}
				\end{itemize}
			}
			{
				\begin{enumerate}[noitemsep]
					\item Uratować wszystkie dobre istoty, w tajemnicy przed Inkwizycją i Watykanem.
					\item Nawrócić złe istoty na dobrą drogę.
					\item Przekonać Ludwika, aby sobie poszedł lub do was dołączył.
					\item Zniszczyć zamek.
				\end{enumerate}

			}{
			\larpcardlineattributes{4}{6}{3}}
			{}

%TODO





% Horacy, wampir, Tzimisce, agent Hadeona Jarosławicza
% Jesteś agentem wysłanym przez samego Hadeona w celu zbadania sprawy zamku i rozszerzenia ewentualnych włości. Wiesz, że zamek jest wielkim Kniaziem stworzonym dawno temu przez potęńnego wojewodę z twojego klanu. Rozpoznajesz moce transformacji i typy spotkanych Szlachciców. Wiesz, że zamek posiada rdzeń, ale nie wiesz gdzie. Posiadasz fiolkę z krwią Hadeona, którą musisz pokropić rdzeń. Potrafisz zmienić swój i czyjś kształt, aby wyglądał na innego lub aby modyfikować jego anatomię. Jeśli nie uda ci się pokropić krwią Hadeona rdzenia, powinieneś napoić nią któregoś wampira, aby był jego sługą.
% Transformacja - (4)
% Przeżyć
% Dać Hadeonowi kontrolę nad zamkiem
% Spowodować wypicie krwi Hadeona przez jakiegoś ważnego wampira
% Kompulcja: Chęć posiadania najcenniejszych rzeczy w zamku

% Hans, wampir, Brujah, agent Księcia Szczecina
% Książę Szczecina jest córką Hadeona, a ty jesteś z nią związany więzami krwi. Zostałeś wysłany tutaj, aby przejąć władzę i poszerzyć domenę.​​​​​​​ Nie wiesz, czemu cię wysłali tutaj, ale masz zrobić co do ciebie należy. Umiesz rozpoznać moce Transformacji, jeśli je zobaczysz. Twój Książę także chciałby od ciebie ochrony którychkolwiek Tzimisce i użytkowników Transformacji.
% Rozprawić się z Camarillą
% Ochronić Horacego
% Kompulsja: Bunt przeciwko narzuconym zasadom

% Bonifacy, wampir, Toreador, agent Księcia Warszawy, Camarilla
% Hania Buszek bardzo by chciała upewnić się, że włości Camarilli zostaną w dobrych rękach.​​​​​​ Słyszałeś, że prócz syna księcia także inni sobie ostrzą kły na ową posiadłość. Zamek powinien być w niezniszczonym stanie, aby nowi lokatorzy mogli bezpiecznie w nim przebywać.
% Rozprawić się z kimkolwiek spoza Camarilli, kto chce przejąć zamek
% Ochronić mury zamku, aby był bezpieczny dla nowych lokatorów
% Kompulsja: Obsesja na punkcie czegoś


% Bartłomiej Bratowicz, naturalny ghul, zły
% Naturalny ghul, którego ród brata się z Sabatem i wszystkim co najgorsze. Bezwzględny i obleśny. Chcesz się zemścić na Duczeskich za odwrócenie się od Tzimisce oraz na podwładnych Hadeona za odwrócenie się od ciebie. Posiadasz branie dystansowe i białe bronie na wampiry. Masz obleśny koc, którym się ubierasz, którego nawet pożar nie chce trawić.
% Gaszenie pożaru (5)
% Przeżyć
% Pokropić rdzeń własną krwią
% Znaleźć sobie jakiegoś poddanego albo kompana do rządzenia
% Zabić wszystkich innych w najokrutniejszy sposób
% Doprowadzić kogoś do załamania nerwowego



% Siostra Anastazja, człowiek, siostra agentka z Watykanu, ludzie
% Inkwizycja nie ma najlepszej renomy w Watykanie. Dlatego sam papież posłał ciebie, abyś kontrolowała postępy w niszczeniu zamku. Posiadasz pistolet na wodę święconą i duże baniaki na plecach. Jesteś bezwzględna w wykonywaniu swojej pracy i nie masz litości dla pomiotów szatańskości. Przypałętał się do was dziwny człowiek Ludwik, nie pozwól aby dołączył i wzmocnił Inkwizycję.
% Gaszenie pożaru (4) - możesz opsikać dostatecznie mały ogień, aby go ugasić
% Upewnić się, że Ojciec Arnold wykonuje poprawnie swoją pracę
% Zniszczyć zamek
% Przekonać Ludwika, aby sobie poszedł i nie dołączał do Inkwizycji
% Zabić wszystkich nieludzi

% Szlachcic Alfa, ghul, stworzony przez Zamek
% Zamek jest potężnym Kniaziem. Tak potężnym, że stworzył ciebie, abyś usunął z ich rejonów wszystkich. Masz zaawansowane moce transformacji, które pozwalają ci przybierać wygląd normalnego wampira. Wiesz, że rdzeń, który steruje zamkiem, został wtopiony w ciało kapitana, a on o tym nie wie. Powinieneś go chronić tak, aby nikt nie widział. Możesz zapytać się Mistrzów Larpa o wszystkie informacje na temat zamku jak miejsca pożarów i zawałów. Znasz pozycje tajnych przejść. Zamek nie atakuje ciebie i nie musisz odkrywać karteczek (ale niech inni o tym nie wiedzą). Gdy zamek zostanie przez kogoś zraniony, możesz napić się z rany jego krwi, aby odzyskać jej punkty. Wiesz, że Książę został porwany i zaabsorbowany przez zamek i poszukiwania jego są daremne. Jeśli ktoś leży martwy lub w letargu, możesz wskazać zamkowi jego pozycję, aby go zaabsorbował. Wtedy zmienia się w twojego pomocnika i może ci pomagać. Jednak nie ma ludzkiego wyglądu, traci wszystkie moce i zachowuje cechy.
% Wskazanie na absorpcję (8), daje ilość punktów życia, ile masz sukcesów
% Chronić kapitana tak, aby nie wzbudzać podejrzeń
% Dać zaabsorbować wszystkich innych, a na końcu kapitana

% Duch przyzwany przez Księcia
% Jesteś duchem snującym się po zamku. Książę przyzwał cię w czasie jednego ze swoich eksperymentów. Jesteś domyślnie niewidoczny dla wszystkich innych, chyba że zdecydujesz się pokazać. Możesz przejąć kontrolę nad kimś, aby chodził razem z tobą i wykonywał twoje rozkazy, używasz do tego czapeczki ze sznurkiem. Znasz pozycje tajnych przejść. Zamek także cię atakuje, ale poważne obrażenia są dla ciebie powierzchowne, a powierzchowne żadne. Nie możesz atakować nikogo żywego, a nieżywych atakujesz mentalnymi cechami i zadajesz obrażenia Siły Woli. Wiesz, że zamek jest potężnym Kniaziem. Jeśli znajdziesz kogoś martwego lub w letargu, możesz wyrwać mu duszę, dzięki czemu stanie się duchem podobnym do ciebie. Traci swoje moce i cechy fizyczne, ale nie musi chcieć ci pomagać.
% Przejęcie kontroli, przeciwko SW (5)
% Gadanie w myślach, przeciwko umysłowym (3)
% Wyrwanie duszy (5) - nowy duch dostaje ilość życia w liczbie punktów sukcesu
% Zemścić się na Księciu
% Stworzyć sobie kompanów do mieszkania razem
% Pozbyć się wszystkich innych z zamku
% Zabić zamek i zostawić mury, albo nawiązać z nim pokój

% Obrońca Włości, potwór, przyzwany przez Księcia, poddany Księcia
% Jesteś wielkim potworem o 4 łapach i paszczy. Zostałeś przyzwany przez Księcia abyś strzegł jego domeny. Lubisz łaskotki po brzuszku. Masz 5 kolców, które możesz wystrzelić. Masz 5 jaj, które możesz w kimś złożyć. Możesz kogoś pożreć, aby zabrać mu jeden punkt życia lub krwi. Za punkt krwi możesz zregenerować kolec. Trawienie kogoś trwa długo, a twoja ofiara pozostaje świadoma. Po przekazaniu jednego życia lub krwi może próbować się wydostać. Jeśli znajdziesz kogoś martwego lub w letargu, możesz w nim złożyć jaja, aby po krótkiej chwili zmienił się w twojego całkowitego poddanego.
% Pożarcie kogoś (7) - przeciwko fizycznym
% Zatrzymanie kogoś w brzuchu (5) - przeciwko fizycznym
% Złożenie jaj (7) - dostaje tyle punktów życia, ile masz sukcesów
% Uratować Księcia
% Zabić zamek
% Pozbyć się wszystkich spoza poddanych Księcia

% Otto, wampir, Gangriel, ogrodnik Księcia
% Chcesz sadzić roślinki i tylko to cię interesuje, Niech wszyscy spadają. Możesz się zmieniać w małe zwierzę i przechodzić przez gruzowiska oraz tajne przejścia, jeśli zobaczysz jak ktoś go używa. Posiadasz bogaty zestaw cisowych kołków, nadają się rarówno do podtrzymywania pomidorów, jak i do wbijania pijawkom w serca. 
% Wypierdolić wszystkich poza zamek
% Przyciąć żywopłot
% Kompulsja: dzikość, drapanie i gryzienie, tylko pojedyncze słowa

% Ludwik, człowiek, fan wampirów, ludzie
% Czytałeś o wampirach i innych nadprzyrodzonych istotach tak dużo, że zawsze chciałeś się nim stać. Przy okazji jesteś ciekawy zamku. Masz sprzęt, który wydaje ci się, że tylko ty umiesz obsługiwać i który pozwala na określenie jednej z pozycji rdzenia. Masz broń przeciwko wampirom, która zadaje silne obrażenia. Chcesz dać się spokrewnić przez wampira, zmienić w ducha itp. Jesteś trochę nieufny wobec nich, żeby nie dać się zwyczajnie wyssać od razu. Inkwizycja bardzo się tobą interesuje.
% Przeżyć
% Przekonać kogoś aby zmienił cię w istotę nadprzyrodzoną lub dołączyć do jakiejś fajnej grupy

% Sebastian, wampir, Brujah, Anarchiści
% Byłeś dobrym kolegą Księcia mimo że on należał do Camarilli. Przyszedłeś go odwiedzić, aby sprawdzić jak się miewa. Uchodzisz wśród innych poddanych za durnego dresa i trochę nim jesteś. Ale za to potrafisz szybko biegać i się wspinać. Jebać Inkwizycję na 100%. Jesteś waleczny, odważny i skłonny do układów. Potrafisz rozniecać pożary, ale za każdym razem testujesz, czy nie uciekniesz ze strachu.
% Wzniecanie pożaru (6) - siła pożaru to liczba sukcesów
% Uratować Księcia
% Przejąć kontrolę nad zamkiem, jakoś
% Kompulcja: bunt przeciwko swoim celom i narzucaniu woli przez innych
